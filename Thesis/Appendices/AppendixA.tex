% Appendix A

\chapter{Übersicht über Trainingsparameter der Wortkontextvektoren} % Main appendix title

\label{AppendixA} % For referencing this appendix elsewhere, use \ref{AppendixA}

\begin{figure}[h]
\centering
\begin{tabular}{c||l|l|l|l|l}
  \textsc{Nr.} & \textsc{Korpus} & \textsc{Prep} & \textsc{Training} & \textsc{NEG} & \textsc{Sampling} \\
  \hline \hline
  \Romannum{1} & Decow & - & Skip-gram & 5 & $1^-5$ \\
  \hline
  \Romannum{2} & Decow & - & Skip-gram & 5 & $1^-4$ \\
  \hline
  \Romannum{3} & Decow & - & Skip-gram & 5 & $1^-3$ \\
  \hline
  \Romannum{4} & Decow & - & Skip-gram & 5 & $0,01$ \\
  \hline
  \Romannum{5} & Decow & - & Skip-gram & 5 & $0,1$ \\
  \hline
  \Romannum{6} & Decow & - & Skip-gram & 5 & $1$ \\
  \hline
  \Romannum{7} & Decow & - & CBOW & 5 & $1^-5$ \\
  \hline
  \Romannum{8} & Decow & - & CBOW & 5 & $1^-4$ \\
  \hline
  \Romannum{9} & Decow & - & CBOW & 5 & $1^-3$ \\
  \hline
  \Romannum{10} & Decow & - & CBOW & 5 & $0,01$ \\
  \hline
  \Romannum{11} & Decow & - & CBOW & 5 & $0,1$ \\
  \hline
  \Romannum{12} & Decow & - & CBOW & 5 & $1$ \\
  \hline
  \Romannum{13} & Decow & Lemmatisiert & Skip-gram & 5 & $1^-5$ \\
  \hline
  \Romannum{14} & Decow & Lemmatisiert & Skip-gram & 5 & $1^-4$ \\
  \hline
  \Romannum{15} & Decow & Lemmatisiert & Skip-gram & 5 & $1^-3$ \\
  \hline
  \Romannum{16} & Decow & Lemmatisiert & Skip-gram & 5 & $0,01$ \\
  \hline
  \Romannum{17} & Decow & Lemmatisiert & Skip-gram & 5 & $0,1$ \\
  \hline
  \Romannum{18} & Decow & Lemmatisiert & Skip-gram & 5 & $1$ \\
  \hline
  \Romannum{19} & Decow & Lemmatisiert & CBOW & 5 & $1^-5$ \\
  \hline
  \Romannum{20} & Decow & Lemmatisiert & CBOW & 5 & $1^-4$ \\
  \hline
  \Romannum{21} & Decow & Lemmatisiert & CBOW & 5 & $1^-3$ \\
  \hline
  \Romannum{22} & Decow & Lemmatisiert & CBOW & 5 & $0,01$ \\
  \hline
  \Romannum{23} & Decow & Lemmatisiert & CBOW & 5 & $0,1$ \\
  \hline
  \Romannum{24} & Decow & Lemmatisiert & CBOW & 5 & $1$ \\
\end{tabular}
\caption[Trainingsparameter der Wortkontextvektoren]{Quelle und Trainingsparameter für verschiedenen Sets von Wortkontextvektoren.
\textsc{Prep} = Aufbereitung des Korpus vor dem Training; \textsc{Training} = Verwendete Trainingsmethode; \textsc{NEG} = Anzahl der
Negativbeispiele beim Training; \textsc{Sampling} = Ausmaß des Downsamplings häufiger Wörter.}
\end{figure}
