% Chapter 7

\chapter{Experiment B} % Main chapter title

\label{Chapter7} % For referencing the chapter elsewhere, use \ref{Chapter1}

%----------------------------------------------------------------------------------------

In diesem Kapitel soll der Ansatz von (\cite{bordes2013translating}) für deutsche Daten nachvollzogen werden.
Dabei wird erst erklärt, wie die Daten erstellt wurden. Danach wird die Idee zum Training der Daten ausgeführt und
auf die neuen Datensätze angewendet, bevor schlußendlich eine Evaluation und eine Gegenüberstellung zu den Originaldaten
erfolgt.

\section{Datenerzeugung}

In (\cite{bordes2013translating}) werden mehrere Datensets erstellt darunter eines namens \emph{FB15k}. Dieses
besteht aus Relationstripeln der Form $(h, l, t)$
(= $(head,\ link,\ tail)$). Diese stammen aus \emph{Freebase}, einer
community-gepflegten Datenbank, in der mehr als 23 Millionen Entitäten durch Relationen miteinander verknüpft sind.
Mittlerweile ist die Seite offline; das Projekt wurde sukzessive in \emph{Wikidata}
\footnote{Siehe \url{https://www.wikidata.org/wiki/Wikidata:Main_Page (zuletzt abgerufen am 20.05.16)}} integriert. Auch die
Freebase API, die als Programmierschnittstelle zum Abfragen von Informationen dient wird langsam abgeschaltet
\footnote{Siehe \url{https://en.wikipedia.org/wiki/Freebase (zuletzt abgerufen am (20.05.16))}}.\\

Die FB15k-Daten enthalten 592.213 Tripeln mit 14.951 einzigartigen Entitäten und 1.345 einzigartigen Relationen.
In Freebase sind Entitäten sprachlich unabhängig gehalten. So wird die Entität mit dem Kürzel ``/m/02vk52z''
im Englischen mit dem Begriff \emph{World Bank} und im Deutschen mit \emph{Weltbank} bezeichnet. Somit sind ist auch
FB15k zumindest theoretisch vielsprachig. Jedoch sind die Entitäten darin oft hauptsächlich englische bzw. amerikanische
Entitäten, die im deutschen Sprachraum teils nicht sehr bekannt sind. Das lässt daraus schließen, dass das Attribut einer
Entität in Freebase, dass den deutschen Namen enthält, nicht immer verwendet wurde. Dies könnte drei Gründ haben:

\begin{enumerate}
  \item Es gibt keine deutsche Übersetzung
  \item Die Entität ist für den deutschen Sprachraum nicht relevant genug
  \item Bisher hat einfach noch kein Nutzer einen deutschen Begriff hinzugefügt
\end{enumerate}

Die Plausibilität dieser Gründe ist diskutabel: Nicht alle Begriffe brauchen eine Übersetzung, so sind beispielsweise
Personennamen i.d.R. durch alle Sprachen hinweg gleich. Schwieriger wird es bei Ortsnamen oder Namen von Organisationen:
Intuitiv drängt sich der Anschein auf, dass populäre Bezeichnungen eher übersetzt werden als unpopulärere
(\emph{United States of America} $\rightarrow$ \emph{Vereinigte Staaten von Amerika} / \emph{University of Denver}
$\rightarrow$ \emph{University of Denver}).\\
