% Chapter 8

\chapter{Experiment C: Relationsvorhersage mit Wortvektorrepräsentation} % Main chapter title

\label{Chapter8} % For referencing the chapter elsewhere, use \ref{Chapter1}

%----------------------------------------------------------------------------------------

In diesem Kapitel soll versucht werden, die Methodik des Rausch-konstrastierenden Lernen aus Kapitel \ref{Chapter6}
auf die Wortvektoren (vgl. Kapitel \ref{Chapter5}) anzuwenden und nur die Relationsvektoren zu lernen. Zu diesem Zweck
wird eine Untermenge von FB15k, die Tripel enthält, deren Kopf- und Fußentität jeweils als Wortvektorrepräsentation vorliegen.
Schlußendlich werden die Ergebnisse evaluiert und diskutiert.

\section{Datenerzeugung}

Zu Trainingszwecken wird eine Untermenge aus FB15k bzw. GER14k gebildet. Da
die Vektorrepräsentationen für die Entitäten nicht von Grund auf erstellt sorden mit
\verb|word2vec| trainierte Vektoren verwendert werden, können nur solche Worte in dem
neuen Trainingsset vorhanden sein, zu denen ebensolche Wortvektoren auch vorliegen.
Dies ist leider nur für einen gerinen Teil der Fall, wie in Abbildung \ref{fig:we4k} zu
erkennen ist.

\begin{figure}[h]
  \centering
  \begin{changemargin}{-1cm}{0cm}
  \resizebox{1.15\textwidth}{!}) & 14.334 & (\textcolor{BrickRed}{$\downarrow$-4,12 \%}) & 1.236 & (\textcolor{BrickRed}{$\downarrow$-8,1 \%}) \\
    WE4k & 50.518 & (\textcolor{BrickRed}{$\downarrow$-91,47 \%}) & 3.623 & (\textcolor{BrickRed}{$\downarrow$-74,72 \%}) & 749 & (\textcolor{BrickRed}{$\downarrow$-39,40 \%}) \\
  \end{tabular}%
  }
\end{changemargin}
  \caption[Daten des neuen Relationsdatensets im Vergleich zu FB15k und GER14k]{Daten des neuen Sets WE4k im Vergleich mit
  FB15k und GER14k. Aufgelistet ist die Anzahl der Tripel (Datensätze), Entitäts- und Relationstypen und die Veränderung
  der Anzahlen in Prozent im Vergleich zu FB15k in Klammern.\label{fig:we4k}}
\end{figure}

Dies ist dadurch zu erklären, dass in der Quelle der Originaldaten, nämlich Freebase,
sehr viele sehr seltene Entitäten vertreten sind. Die vielen Freiwilligen, die über die
Jahre dazu beigetragen haben, die Wissensdatenbank weiter zu vervollständigen, haben dabei auch
viele unbekanntere Filme, Personen etc. hinzugefügt, die in einem Korpus eher selten zu finden sind.\\

Da die resultierende Menge an Tripeln gerundet ungefähr viertausend einzigartig Entitäten enthält, die auch als
Wortvektorrepräsentation (\emph{word embeddings}) vorliegen, wird dieses Datenset WE4k genannt.

\section{Training}

Das Training der Relationsvektoren findet in einer sehr zu der in Kapitel \ref{Chapter6} beschrieben Methode sehr
ähnlichen Art und Weise wie in Kapitel \ref{Chapter6} statt: Es wird zuerst eine Menge korrumpierter Tripel erstellt, bei denen entweder
die Kopf- oder Fußentität ersetzt wurde.
Mithilfe der ursprünglichen Tripel wird eine Menge aus Tripelpaaren erstellt, von denen der korrumpierte Teil jeweils
zufällig aus $S'$ ausgewählt wird. Auch die Verlustfunktion gestaltet sich prinzipiell gleich.\\

Der einzige Unterschied besteht in der Implementation des Algorithmus: So wird wird wegen der geringeren Menge an
Daten darauf verzichtet, für jede Trainingsanzahl eine feste Anzahl von Tripelpaaren zufällig auszuwählen.
Stattdessen richtet sich die Größe von $S_{batch}$ nach folgender Formel:

\begin{equation}
  S_{batch} \leftarrow sample(S^*, \floor*{\frac{|S^*|}{10}}+1)\footnote{Die Addtion von 1 dient dazu, bei weniger
  von 10 Tripeln für eine Relation wenigstens mit einem Tripel zu trainieren. Der Rest ist eine Heuristik.}
\end{equation}



\section{Evaluation}

\begin{figure}[h]
  \centering
  \begin{tabular}{r||c|c}
    \textsc{Datenset} & \textsc{Gemittelter Rang} & \textsc{Hits@10} \\
     \hline
     WE4k & 1122,03 & 5,67 \\
  \end{tabular}
  \caption[Resultate auf mit Wordvektoren auf WE3k]{}
\end{figure}
