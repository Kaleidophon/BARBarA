% Chapter 4

\chapter{Vorbereitung} % Main chapter title

\label{Chapter4} % For referencing the chapter elsewhere, use \ref{Chapter1}

%----------------------------------------------------------------------------------------

\section{Vorbereitung}

  Für die nachfolgenden Experimente in den Kapiteln \ref{Chapter5}, \ref{Chapter6}, \ref{Chapter7} und \ref{Chapter8}
  mussten einige Daten zuerst beschafft und unter Umständen aufbereitet werden. Die ergriffenen Maßnahmen werden
  deshalb in diesem Kapitel beschrieben.

  \subsection{Verwendete Daten}

  Die anfangs verwendeten Daten lassen sich grob in vier Kategorien einteilen: Den Korpus, das Relationsdatenset,
  Wortähnlichkeitslisten sowie Analogiedatensets (siehe Abbildung \ref{fig:datasets}).\\
  Als Korpus wurde der DECOW14X-Korpus verwendet. Genaueres über dessen Beschaffenheit und Vorverarbeitung für das
  Training von Wortvektoren werden in \ref{sec:corpusprep} und \ref{sec:vectrain} beschrieben.

  \begin{figure}[h]
    \centering
    \def\arraystretch{1.5}
    \resizebox{\textwidth}{!}{%
    \begin{tabular}{l|l|rl|l}
      \textbf{Art} & \textbf{Name} & \multicolumn{2}{l|}{\textbf{Beschreibung}} & \textbf{Quelle} \\
      \hline \hline
      Korpora & \textsc{DECOW14X} & \multicolumn{2}{l|}{Deutschsprachiger Webkorpus} & (\cite{schafer2012building}) \\
      \hline
      Relationsdaten & \textsc{FB14k} & \multicolumn{2}{l|}{592k Relationstripel aus Freebase} & (\cite{bordes2013translating}) \\
      \hline
      \multirow{4}{*}{\parbox[t]{1.5cm}{Wortähnlich-keiten}} & \textsc{Schm280} & 280 & \multirow{4}{*}{\Vast \} \parbox[c]{4cm}{nach Ähnlichkeit bewertete Wortpaare}} & (\cite{koper2015multilingual})\\ \cline{5-5}
       & \textsc{Wortpaare65} & 65 & & \multirow{3}{*}{(\cite{rubenstein1965contextual})} \\
       & \textsc{Wortpaare222} & 222 & & \\
       & \textsc{Wortpaare350} & 350 & & \\
      \hline
      \multirow{2}{*}{Analogien} & \textsc{Google} & \multicolumn{2}{l|}{Datenset mit 18.552 Analogien} & \multirow{2}{*}{(\cite{koper2015multilingual})} \\
       & \textsc{SemRel} & \multicolumn{2}{l|}{Datenset mit 2.462 Analogien} & \\
    \end{tabular}%
    }
  \caption[Übersicht über verwendete vorgefertigte Datensets]{Übersicht über in dieser Arbeit verwendete, extern generierte
  Datensets sowie deren Quelle.\label{fig:datasets}}
  \end{figure}

  Relationsdaten aus \textsc{Freebase} wurden im Datenset \textsc{FB14k} von (\cite{bordes2013translating}) gesammelt.
  Diese werden vor allem in den Versuchen von Kapiteln \ref{Chapter6} und \ref{Chapter8} relevant.\\
  Alle anderen Datensets sind zur Evaluation der Wortvektoren in Kapitel \ref{Chapter5} angedacht: Die Sets
  \textsc{Schm280}, \textsc{Wortpaare65}, \textsc{Wortpaare222} und \textsc{Wortpaare350} bestehen aus einer Reihe
  von Wortpaaren, die von menschlichen Annotatoren auf verschiedenen Skalen nach Ähnlichkeit bewertet wurden.
  \textsc{Google}\footnote{Eigentlich: \textsc{Google semantic/syntactic analogy datasets}} und \textsc{SemRel} sind
  in verschiedene Kategorien eingeteilte Analogiepaare der Form ``W verhält sich zu X wie Y zu Z''. Bei der Evaluation
  wird dann die Fähigkeit eines Systems mit Wortvektoren getestet, diese korrekt zu vorvervollständigen, wenn Z nicht
  gegeben ist.


  \subsection{Aufbereitung des Korpus}\label{sec:corpusprep}

  Als Textressource wurde das DECOW14X-Korpus (DE = Deutsch, COW = ``\textbf{CO}rpus from the \textbf{W}eb'') verwendet.
  Dieses Korpus von (\cite{schafer2012building})  besteht aus 21 Texten,
  die in den Jahren 2011 und 2014 von deutschsprachigen Internetseiten gecrawled und aufbereitet wurden. Dies beinhaltet
  PoS-Tagging, Chunking, Lemmatisierung, das Markieren von Eigennamen (Named Entities) und dem Hinzufügen von Metadaten.
  Die Sätze liegen darin im CoNLL-Format\footnote{Siehe \url{http://ilk.uvt.nl/conll/} (zuletzt abgerufen am 11.04.16)} vor,
  wobei jedem Wort und dessen Annotationen eine ganze Zeile gewidmet ist,
  Stazgrenzen werden durch XML-Tags getrennt. Summa summarum enthält das Korpus 624.767.747 Sätze mit 11.660.894.000 Tokens.\\

  Für diese Arbeit wurden auf Basis der Ressource drei Version für das Training der Wortvektoren erstellt:
  \begin{itemize}
      \item Eine Datei mit den originalen Tokens durch Leerzeichen getrennt, je ein Satz pro Zeile.
      \item Eine Datei mit den lemmatisierten Tokens durch Leerzeichen getrennt, je ein Satz pro Zeile.
  \end{itemize}


  \subsection{Training der Wortvektoren}\label{sec:vectrain}

  Wortvektoren werden mithilfe des Tools \emph{word2vec} und zwei verschiedenen Modellen trainiert: Continuous-Bag-of-Words (CBOW)
  und Skip-Gram. Das CBOW-Model wurde zuerst von (\cite{mikolov2013efficient}) vorgestellt. Die Erklärung der Funktionsweise
  wird im nachfolgenden Teil recht klein gehalten, für eine ausführlichere und verständliche Ausführung wird beispielsweise
  die Arbeit von (\cite{rong2014word2vec}) empfohlen.\\

  Zum Training der Vektoren wurde das C-Tool \emph{Word2Vec} von (\citeauthor{mikolov2013efficient}) verwendet. Als Eingabe
  benötigt es eine Textressource, die einen Satz pro Zeile enthält, Tokens durch Leerzeichen getrennt und gibt die Wortvektoren
  entweder einem einfach Text- oder Binärformat aus.\\
  Das Tool lässt zudem dem Nutzer offen, einige Parameter zu verändern. Jene, die in dieser Arbeit berücksichtigt wurden, sollen
  dabei näher erläutert werden:
  \begin{itemize}
    \item \verb|-sample|\\Die Wahrscheinlichkeit, mit der hochfrequente Worte
    \item \verb|-cbow|\\Bestimmt, welche Trainingsmethode verwendet wird ($0\ \hat{=}$ Skip-gram, $1\ \hat{=}$ Continuous-Bag-of-Words)
    \item \verb|-negative|\\Anzahl von negativen Beispielen beim Training.
  \end{itemize}

  Zwar bietet das Tool auch noch andere Parameter, jedoch soll aufgrund mit der Empfehlungen in (\citeauthor{levy2015improving}), in
  der eine große Anzahl von Konfigurationen ausprobiert wurde, im Rahmen dieser Arbeit nur mit den oben genannten Werten experimentiert werden.
