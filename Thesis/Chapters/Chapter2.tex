% Chapter 2

\chapter{Verwandte Arbeiten} % Main chapter title

\label{Chapter2} % For referencing the chapter elsewhere, use \ref{Chapter1}

%----------------------------------------------------------------------------------------

\section{Wortvektorrepräsentationen}

\emph{Word embeddings} (=``Worteinbettungen'') wurden zuerst von (\cite{bengio2006neural}) noch vor dem erneuten
Deep-Learning-Boom 2006 präsentiert. Im Vergleich zu vorherigen Arbeiten mit sog. ``One-Hot''-Vektoren, bei denen
jede Dimension einem Wort des Vokabulars eines Korpus zugeordnet ist und nur eine davon den Wert $1$ besitzt (gewöhnlich
besitzen alle anderen den Wert $0$), sind Informationen über ein Wort in dieser Repräsentation für den Menschen uneindeutig
und kontinuerlich über den Vektor verteilt (``distributed represenation''), woher sich auch der Name des dazugehörigen
Forschungsfeldes - der distributionellen Semantik - herleitet. Dies ist aber eigentlich nur die Folge eines viel
grundlegeren Unterschied der Methodik, der von (\cite{baroni2014don}) weiter ausgeführt: Lange hatte man sich
mit zählbasierten Methoden beschäftigt, um solche Repräsentationen zu erstellen; mit (\cite{bengio2006neural}) hielten
vorhersagebasierte Methoden schließlich Einzug. \\
Diese fußt auf der \emph{Distributional hypthesis}, die besagt, dass Worte, die in ähnlichen Kontexten auftauchen, dazu tendieren, eine ähnliche Bedeutung zu haben (\cite{harris1954distributional}).
Diese Besonderheit macht sich (\cite{bengio2006neural}) zunutze, da das neurale Netz, mit denen die Wortvektorrepräsenationen
trainiert werden, sich beim Bewegen durch einen Trainingskorpus ein mehrwortiges Kontextfenster um ein Zielwort herum
verwendet. Das System lernt, die Wortvektoren an häufig vorkommende Wortkontexte anzupassen. Selbige besitzen zudem
weitere nützliche Eigenschaften, die in Kapitel \ref{Chapter3} beschrieben werden.\\
Diese Forschungsarbeit löste eine Welle weiterer Forschungen in diesem Gebiet aus, da gezeigt werden konnte, dass diese
Art der Wortrepräsentationen genutzt werden konnte, die Leistungen von Systemen selbst im Bezug auf lange bekannte und
erforschte Probleme der Computerlinguistik signifikant verbessern konnte. Exemplarisch sei hier die Arbeit von
(\cite{collobert2011natural}) genannt, die so neue Ansätze für PoS-Tagging, Named-Entity-Recognition, Chunking und Semantic
Role Labeling präsentieren.\\

Weitere Untersuchungen beschäftigen sich u. A. mit bilingualen Repräsentationen für maschinelle Übersetzung (\cite{zou2013bilingual}),
deren Training auf der Basis von einem dependenzgeparsten Korpus (\cite{levy2014dependency}), dem Optimieren der Parameter
(\cite{levy2015improving}) u.v.m.


\section{Vektorrepräsentationen für Wissensdatenbank}

Entitäten und Relationen aus Wissensdatenbanken mithilfe eigener Vektoren zu respräsentieren wurde bereits von einigen
Forschungsarbeiten in Angriff genommen: Als einer der ersten Ansätze versuchen (\cite{bordes2011learning}), symbolische
Wissensrepräsentationen in einen Vektorraum einzubetten und so für die Künstliche-Intelligenz-Forschung leichter nutzbar
zu machen. Diese sog. \emph{Structured Embeddings} (SE) werden mithilfe von neuralen Netzwerken trainiert.
Darauf aufbauend präsentieren (\cite{bordes2013translating}) einen etwas simpleren Ansatz namens \emph{TransE}, der darauf abzieht, Relationen
zwischen Entitäten als eine einfach vektorarithmetische Operation (= Übersetzung) zu sehen. Diese Vorgehensweise wird in Kapitel \ref{Chapter6}
etwas genauer ausgeführt und dere Ergebnisse für einen deutschsprachigen Datensatz repliziert.\\

\begin{figure}[h]
  \centering
  \bgroup
  \def\arraystretch{1.5}
  \resizebox{0.8\columnwidth}{!}{%
  \begin{tabular}{l|l}
    \textsc{Modell} & \textsc{Scoring-Funktion} \\
    \hline \hline
    \emph{TransE} & $f_r(h, t) = \parallel h + r - t \parallel^2_2$ \\
    \hline
    \emph{TransH} & $f_r(h, t) = \parallel h_{\bot} + r - t_{\bot} \parallel^2_2$ \\
    \hline
    \multirow{3}{*}{\emph{TransR}} & $f_r(h, t) = \parallel h_r + r - t_r \parallel^2_2$ \\
     & $h_r = hM_r$ \\
     & $t_r = tM_r$ \\
    \hline
    \multirow{3}{*}{\emph{CTransR}} & $f_r(h, t) = \parallel h_{r, c} + r_c - t_{r, c} \parallel^2_2 + \alpha\parallel r_c-r\parallel^2_2$ \\
     & $h_{r,c} = hM_r$ \\
     & $t_{r,c} = tM_r$ \\
  \end{tabular}%
  }
  \egroup
   \\\vspace{0.5cm}

\textbf{Erklärung der Parameter}

\begin{multicols}{2}

  \begin{itemize}
    \item $(h, r, t)$: Relationstripel bestehend aus einer Kopf- ($h$) und Fußentität ($t$) sowie einer Relation $r$
    \item $f_r(\cdot, \cdot)$: Scoringfunktion zweier Entitäten für eine Relation $r$
    \item $h_{\bot}, t_{\bot}$: Projektionen zweiter Entitätsvektoren auf eine Ebene
  \end{itemize}

  \columnbreak

  \begin{itemize}
    \item $M$: Projektionsmatrix
    \item $(h_{r, c}, r_c, t_{r, c})$: Relationstripel mit auf ein Subcluster einer Relation trainieren Vektorrepräsentationen
    \item $\alpha$: Gewichtsparameter zur Einschränkung für den Abstand einer Subrelation zur Hauptrelation
  \end{itemize}

  \end{multicols}
  \caption[Übersicht über verschiedene Arten der Vektorrepräsentationen für Wissensdatenbanken]{Übersicht über verschiedene Arten der Vektorrepräsentationen für Wissensdatenbanken
  Für TransE (\cite{bordes2013translating}), TransH (\cite{wang2014knowledge}), TransR und CTransR (\cite{lin2015learning})
  werden die verschiedenen Scoringfunktionen gegenübergestellt und vorkommende Parameter erklärt. \label{fig:scoring}}
\end{figure}

Diese Idee wird weiter von (\cite{wang2014knowledge}) vorangetrieben: Um nicht nur 1:1- sondern auch 1:n-, n:1- und m:n-Relationen
abzubilden, werden Punkte in einem Vektorraum auf eine für jede Relation separat gelernte Ebene projiziert, woher sich
der Name des Verfahrens \emph{TransH}, herleitet.\\
Um die Vektorräume von Entitäten und Relationen zu trennen, stellen (\cite{lin2015learning}) \emph{TransR} und \emph{CTransR}
vor. Für Ersteres wird für jede Relation eine Projektionsmatrix gelernt, die Entitäten in den jeweiligen Relationsvektorraum
übersetzt. Bei Letzterem werden für jede Relation mehrere Vektoren trainiert, um der Unterschiedlichkeit im Kontext
anderer Entitäten gerecht zu werden. Eine Übersicht über die Berwertungsfunktionen aller Verfahren findet sich in Abbildung
\ref{fig:scoring}.
