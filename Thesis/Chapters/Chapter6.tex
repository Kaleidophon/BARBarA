% Chapter 6

\chapter{Experiment A} % Main chapter title

\label{Chapter6} % For referencing the chapter elsewhere, use \ref{Chapter1}

%----------------------------------------------------------------------------------------

\begin{itquote}
Q: Why did the multithreaded chicken cross the road?
A: to To other side . get the
\flushright
\textsc{Jason Whittington}
\end{itquote}

\section{Idee}

Gegeben ist ein Vektorraum $V$ mit Wortvektoren $\vec{u}, \vec{v} \in V$ und $\vec{u},\vec{v}\in\mathbb{R}^d$.
Gesucht ist eine Funktion $\phi$, die ein Vektorenpaar in einen Relationsraum abbildet: $\phi: \mathbb{R}^d \to \mathbb{R}^e$, wobei
nicht zwangsläufig $d = e$.\\
In ihrer einfachsten Form bildet sie einfach die Differenz $\vec{d}$ der beiden Vektoren:
\begin{equation}
  \phi(\vec{u}, \vec{v}) = \vec{v} - \vec{u} = \vec{d}
\end{equation}

\section{Algorithmus}

Der Grundalgorithmus (siehe Fig. X) versucht nun, alle Kombinationen von Wortvektoren zu bilden und über diese
und den Differenzvektor zu bilden. Alle Kombinationen würde bei $n$ Vektoren in $n * n = n^2$ Vektoren resultieren.
Zwar ist $\phi(\vec{u}, \vec{v}) \neq \phi(\vec{v}, \vec{u})$, jedoch wäre die Berechnung beider Differenzvektoren redundant,
da sie lediglich Spiegelungen voneinander im Raum sind und so diesselbe Information enthalten. Die Berechnung
der Differenz in nur eine Richtung reduziert die Anzahl der Vektoren dadurch zu $\frac{n * (n-1)}{2}$.

\begin{figure}[h]
  \centering
  \begin{algorithm}[H]
    \KwData{Menge von Vektorpaaren $\mathcal{C}$}
    \For{$(\vec{u}, \vec{v}) \in \mathcal{C}$}{
      $\vec{d} = \vec{v} - \vec{u}$;
    }
  \end{algorithm}
  \caption[Einfacher Projektionsalgorithmus]{Einfacher Projektionsalgorithmus.}
\end{figure}

Gegeben einer Menge relevanter Kombinationen $\mathcal{C}$ mit Vektorpaaren gilt also:
\begin{equation}
  \forall (\vec{u}, \vec{v}) \in \mathcal{C}: (\vec{v}, \vec{u}) \notin \mathcal{C}
\end{equation}

Eine Modifikation des Algorithmus besteht daran, das Berechnen von $\vec{d}$ ab eine Bedingung zu knüpfen.
Gegeben sei eine Menge von Sätzen (ein Korpus) $\mathcal{K}$ mit $n$ Sätzen $s_i$ sodass $\mathcal{K} = \{s_i\}_{i=1}^{n}$ mit
$m$ Wörtern $w_{ij}$ pro Satz $s_i = \{w_{ij}\}_{j=1}^m$. Eine Kookkurrenz von zwei Begriffen (\emph{types}) $t_1$ und $t_2$
besteht demnach, falls im Korpus mindestens ein Satz existiert, in dem beide gleichermaßen vorkommen. Dazu können wir
eine Funktion $\Lambda$ definieren, die die Anzahl der Kookkurrenzen bestimmt:
\begin{equation}
  \Lambda(t_1, t_2) = |\{s | s\in\mathcal{K} \land \exists w_1\in s \land \exists w_2\in s \land w_1 = t_1 \land w_2 = t_2 \land w_1 \neq w_2\}|
\end{equation}
Eine mögliche Einschränkung besteht darin, in einem geänderten Algorithmus (siehe Fig. X) das Berechnen von $\vec{d}$ nur
dann zu erlauben, wenn die Anzahl der Kookkurrenzen der zu den Vektoren $\vec{v}(t_1), \vec{v}(t_2)$ gehörenden Begriffe $t_1, t_2$
einen bestimmten Schwellenwert $\gamma$ überschreitet, also $\Lambda(t_1, t_2) > \gamma$.

\begin{figure}[h]
  \centering
  \begin{algorithm}[H]
    \KwData{Menge von Vektorpaaren $\mathcal{C}$}
    \For{$(\vec{v}(t_1), \vec{v}(t_2)) \in \mathcal{C}$}{
      \If{$\Lambda(t_1, t_2) > \gamma$}{
        $\vec{d} = \vec{v}(t_2) - \vec{v}(t_1)$;
      }
    }
  \end{algorithm}
  \caption[Modifizierter Projektionsalgorithmus]{Modifizierter Projektionsalgorithmus, bei dem die zu den Vektoren gehörigen
  Begriffe über $\gamma$ Mal im Korpus im gleichen Satz aufgetreten sein müssen, damit $\vec{d}$ errechnet wird.}
\end{figure}


\section{Parallelisierter Algorithmus}
