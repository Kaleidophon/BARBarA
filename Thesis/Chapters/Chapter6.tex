% Chapter 6

\chapter{Experiment A} % Main chapter title

\label{Chapter7} % For referencing the chapter elsewhere, use \ref{Chapter1}

%----------------------------------------------------------------------------------------

In diesem Kapitel soll der Ansatz von (\cite{bordes2013translating}) für deutsche Daten nachvollzogen werden.
Dabei wird erst erklärt, wie die Daten erstellt wurden. Danach wird die Idee zum Training der Daten ausgeführt und
auf die neuen Datensätze angewendet, bevor schlußendlich eine Evaluation und eine Gegenüberstellung zu den Originaldaten
erfolgt.

\section{Datenerzeugung}

In (\cite{bordes2013translating}) werden mehrere Datensets erstellt darunter eines namens \emph{FB15k}. Dieses
besteht aus Relationstripeln der Form $(h, l, t)$
(= $(head,\ link,\ tail)$). Diese stammen aus \emph{Freebase}, einer
community-gepflegten Datenbank, in der mehr als 23 Millionen Entitäten durch Relationen miteinander verknüpft sind.
Mittlerweile ist die Seite offline; das Projekt wurde sukzessive in \emph{Wikidata}
\footnote{Siehe \url{https://www.wikidata.org/wiki/Wikidata:Main_Page (zuletzt abgerufen am 20.05.16)}} integriert. Auch die
Freebase API, die als Programmierschnittstelle zum Abfragen von Informationen dient wird langsam abgeschaltet
\footnote{Siehe \url{https://en.wikipedia.org/wiki/Freebase (zuletzt abgerufen am (20.05.16))}}.\\

Die FB15k-Daten enthalten 592.213 Tripeln mit 14.951 einzigartigen Entitäten und 1.345 einzigartigen Relationen.
In Freebase sind Entitäten sprachlich unabhängig gehalten. So wird die Entität mit dem Kürzel ``/m/02vk52z''
im Englischen mit dem Begriff \emph{World Bank} und im Deutschen mit \emph{Weltbank} bezeichnet. Somit sind ist auch
FB15k zumindest theoretisch vielsprachig. Jedoch sind die Entitäten darin oft hauptsächlich englische bzw. amerikanische
Entitäten, die im deutschen Sprachraum teils nicht sehr bekannt sind. Das lässt daraus schließen, dass das Attribut einer
Entität in Freebase, dass den deutschen Namen enthält, nicht immer verwendet wurde. Dies könnte drei Gründ haben:

\begin{enumerate}
  \item Es gibt keine deutsche Übersetzung
  \item Die Entität ist für den deutschen Sprachraum nicht relevant genug
  \item Bisher hat einfach noch kein Nutzer einen deutschen Begriff hinzugefügt
\end{enumerate}

Die Plausibilität dieser Gründe ist diskutabel: Nicht alle Begriffe brauchen eine Übersetzung, so sind beispielsweise
Personennamen i.d.R. durch alle Sprachen hinweg gleich. Schwieriger wird es bei Ortsnamen oder Namen von Organisationen:
Intuitiv drängt sich der Anschein auf, dass populäre Bezeichnungen eher übersetzt werden als unpopulärere
(\emph{United States of America} $\rightarrow$ \emph{Vereinigte Staaten von Amerika} / \emph{University of Denver}
$\rightarrow$ \emph{University of Denver}).\\
Im Falle der Relevanz ist davon auszugehen, dass diese mit der Bearbeitung durch Nutzer einher geht: Bei einer großen
Nutzerbasis ist davon auszugehen, dass diese primär Einträge von Entitäten bearbeiten, die im Kontext der Geschichte,
des Tagesgeschehens o. Ä. relevant sind. Gegeben genug Zeit und aktive Nutzer ist also anzunehmen, dass eine immer
größer werdende Prozentzahl von Relevanten Entitäten an das Deutsche angepasst wird. Bedenkt man die Laufzeit von Freebase
seit 2007 (also zum Zeitpunkt des Schreibens dieser Arbeit rund 9 Jahre), so nehmen wir an, dass dieses Bedenken zwar nicht
ganz aus der Welt zu räumen, aber zumindest zu vernachlässigen ist.\\

Basierend auf dieser Argumentation werden korrespondierende ``deutsche'' Daten folgendermaßen erzeugt:
Es wird bei allen Relationstripeln eine Prüfung durchgeführt, ob beide Entitäten $h$ und $t$ eine deutsche Bezeichnung
besitzen. Falls nicht, wird dieses Tripel ausgelassen. Danach wird ggf. noch die inverse Relation ergänzt (diese wird
später beim Training benötigt), z.B. bei \emph{/location/location/contains} und \emph{/location/location/containedby}.

\begin{figure}[h]
  \centering
  \begin{tabular}{r||ccc}
    \textsc{Datenset} & \textsc{\#Tripel} & \textsc{\#Entitätstypen} & \textsc{\#Relationstypen} \\
    \hline
    FB15k & 592.213 & 14.951 & 1.345 \\
    GER14k & 459.724 & 14.334 & 1.236 \\
  \end{tabular}
  \caption[Daten über die Relationsdatensets FB15k und GER14k]{Daten über die Relationsdatensets FB15k und GER19k.
  Aufgelistet ist die Anzahl der Tripel (Datensätze), Entitäts- und Relationstypen.}
\end{figure}

Somit gilt für die Menge englischer Relationstripel $S_{EN}$ und die Menge deutscher Tripel $S_{DE}$, dass letztere
eine echte Teilmenge ist: $S_{EN} \supsetneq S_{DE}$ \footnote{$\forall e \in S_{DE}: e \in S_{EN} \wedge S_{DE} \neq S_{EN}$}.

\section{Training}

Gegeben ist eine Menge von Relationstripeln $S$ der Form $(h, l, t) \in S$. Zusätzliche gibt es noch eine Entitätsmenge
$E$ sodass und $h, t \in E$ und eine Menge von Relationen $L$ mit $h \in L$. Den Entitäten und Relationen werden zuästzlich
noch Vektoren aus $\mathbb{R}^k$ zugewiesen. Idealerweise gelten nach dem Training für eine gültige Relation $(h, l, t)$
dann $h + l \approx t$. Hinzu wird noch eine Menge aus korrumpierten bzw. ``falschen'' Tripeln $S'$ erstellt, indem
für jedes Tripel in $S$ entweder $h$ oder $t$ durch eine andere, zufällig gewählte Entität ersetzt wird:
\begin{equation}
  S' = \{(h', l, t) | h' \in E\} \cup \{(h, l, t') | t' \in E\}
\end{equation}

Um die für das Training benötigte Verlustfunktion du definieren wird das Unähnlichkeitsmaß $d_p$ für ein Tripel bestimmt, welcher
entweder aus der $L_1$- oder der $L_2$-Norm abgeleitet wird, also:
\begin{equation}
    d_1(h + l, t) = \sum_{i=1}^k \| h_i + l_i - t_i \|
\end{equation}
\begin{equation}
    d_2(h + l, t) = \sqrt{\sum_{i=1}^k \| h_i + l_i - t_i \|^2}
\end{equation}

Nun kann die Verlustfunktion $\mathcal{L}$ erstellt werden:
\begin{equation}
  \mathcal{L} = \sum_{(h,l,t) \in S} \sum_{(h', l', t') \in S'} max(0, \gamma + d_p(h + l, t) - d(h' + l, t'))
\end{equation}

Der Parameter $\gamma$ bezeichnet hier einen Hyperparameter, der dem System einen gewissen Spielraum lässt. Indem versucht
wird, den durch diese Funktion errechneten Verlust zu minimieren, wird beim Training sichergestellt, dass die Vektorrepräsentationen
von korrekten Tripeln die Gleichung $h + l \approx t$ bestmöglichst erfüllen und für korrumpierte Tripel möglichst weit
verfehlen.\\
Vor dem Training werden die Repräsentationen der Entitäten und Relationen mit einer uniformen Verteilung im Intervall
$[-\frac{6}{\sqrt{k}}, \frac{6}{\sqrt{k}}]$ initialisiert und im Falle von $l \in L$ zusätzlich normiert. Während jedes
Trainingsschritt werden für die aktuellen Repräsentation der Verlust berechnet und deren Werte mithilfe des
Minibatch-Stochastic-Gradient-Descent angepasst. Der Minibatch enthält dabei ebenso viele gültige wie korrumpierte Tripel.
Das Training wird solange ausgeführt, bis die Fehlerrate auf dem Validationsset konvergiert\footnote{Konvergenz auf dem
Trainingsset könnte ein Zeichen für Overfitting sein.}.\\

Für das Reproduzieren der Ergebnisse wurde den Empfehlen von (\cite{bordes2013translating}) gefolgt: So wird das Training
nach maximal 1000 Epochen beendet, außerdem gilt $k = 50$, $\gamma = 1$, $\lambda = 0,01$ und $d_1$ als Unähnlichkeitsmaß.

\section{Evaluation}

Um die erstellten Daten zu evaluieren, werden bei den Relationen im Testset jeweils die Head- und Tailentitäten ersetzt.
Danach werden alle bekannten Entitäten eingesetzt und mithilfe des Dissimilaritätsmaßes ansteigend sortiert. Der Rang
der korrekten (ursprünglich entfernten) Entität wird dann gespeichert. Dieses Verfahren wurde zuerst von (\cite{bordes2011learning})
vorgeschlagen.\\
Daraus lassen sich dann zwei Evaluationsmaße ableiten: Den gemittelten Rang der korrekten Entität sowie die den Anteil
der Evaluationsläufe, bei denen sich die richtige Entität innerhalb der Top 10 befand (Hits@10). Zusätzlich wurde noch
zwischen roh und gefiltert unterschieden: Bei letzterem werden mögliche Tripel ignoriert, die in dieser Form schon in
den Trainingsdaten vorkamen und so auf ``unfaire'' Art und Weise vor allen anderen Tripeln bevorzugt werden.
Die Ergebnisse für die auf FB15k und GER14k trainierten Daten sind in Abbildung X festgehalten.


\begin{figure}[h]
  \centering
  \begin{tabular}{r||cc|cc}
    \multirow{2}{*}{\textsc{Datenset}} & \multicolumn{2}{c|}{\textsc{Gemittelter Rang}} & \multicolumn{2}{c}{\textsc{Hits@10}} \\
     & \emph{Roh} & \emph{Gefiltert} & \emph{Roh} & \emph{Gefiltert} \\
     \hline
     FB15k & 575,54 & \textbf{197,02} & \textbf{34,8} & \textbf{48,79} \\
     GER14k & \textbf{234,73} & 222,05 & \textbf{34,07} & 41,83 \\

  \end{tabular}
  \caption[Resultate für TransE mit FB15k und GER14k]{Resultate für TransE, trainiert auf dem FB15k- und dem GER14k-Datenset.
  Hits\@10 gibt den Anteil der Vorkommen an, bei denen sich die gesuchte Entität in der Relation links oder rechts in den top zehn
  von dem System vorhergesagten Kandidaten befand (\emph{raw}). \emph{Filtered} bezieht sich auf denselben Rang, wenn die extra erstellten
  korrumpierten Tripel zum Training aus den Trainingsdaten entfernt wurden. \emph{mean} bezeichnet den gemittelten Rang der richtigen
  Entität in der vom System erstellten Rangliste der wahrscheinlichsten Vorhersagen.}
\end{figure}

An Ergebnissen ist zuerst zu erkennen, dass sie von den Ergebnissen in (\cite{bordes2013translating}) für FB15k etwas abweichen.
Dies könnte daran liegen, das bei der Beschreibung des Trainingsverfahren angegeben wurde, dass das Trainingsverfahren
nicht unbedingt 1000 Epochen lang ausgeführt wurde. In dem Code, der von den Autoren online zur Verfügung gestellt wurde
\footnote{\url{https://github.com/glorotxa/SME}}, wurde dazu leider keine Option gefunden. Weshalb vermutlich davon auszugehen
ist, dass dies manuell ausgeführt wurde. Deshalb ist zu schließen, dass die hier aufgeführten Ergebnisse schlechter sind als die,
die im Paper angegeben wurde, weil die Vektorrepräsentationen zu sehr auf die Trainingsdaten angepasst wurden und dann schlecht
auf den Testdaten generalisieren. Da die Daten aus GER14k gewissermaßen eine Untermenge von FB15k sind, sind deren Ergebnisse
eher i.d.R. gleichwertig oder schlechter. Die Frage, warum der rohe gemittelte Rang bei GER14k signifikant besser ist als
bei FB15k, könnte dahingehend beantwortet werden, dass sich in der Trainingsphase zu sehr an die Übungsdaten angepasst wurde und
zuviele Trainingstripel vor der eigentlich richtigen Lösung rangieren. Deshalb ist der Wert für die gefilterten Tripel auch umso viel
besser.

[STATISTISCHE SIGNIFIKANZ HITS@10 ROH]

\section{Fazit}

In diesem Kapitel wurde ein Verfahren vorgestellt, dass es erlaubt, Vektoren zu trainieren, die Entitäten und Relationen
aus Wissendatenbanken modellieren und zur Relationsvorhersage zu nutzen. Dabei bestehen die Daten aus Relationstripeln,
also 3-Tupeln mit einer Kopf- und einer Fußentität und einer dazugehörigen semantischen Relation. Es werden außerdem noch
 egativbeispiele erzeugt, indem einer der beiden Entitäten durch eine andere in den Trainingsdaten vorkommende Entität zufällig ausgetauscht wird.
Der Trainingsvorgang wird hierbei mithilfe eines auf Vektorarithmetik basierenden Unähnlichkeitsmaß betrieben, mit der
die ``Stimmigkeit'' korrekter Tripel versucht wird zu maximieren. Das Gegenteil stimmt für korrumpierte Tripel.\\
Es wurde gezeigt, dass sich die (\cite{bordes2011learning}) präsentierten Ergebnisse im Rahmen der gegebenen Möglichkeit
annähernd replizieren lassen. Darüber hinaus wurde die Methodik auch auf eine Menge deutscher Daten angewendet, die eine
echte Teilmenge der englischen Daten darstellt. Für diese konnten ähnlich gute Ergebnisse erziehlt werden, wobei die qua
definition kleinere Menge an Übungsbeispielen dem Unterschied in Performanz Rechnung trägt.\\

Bei dieser Vorgehensweise wurden die Vektorrepräsentationen für Entitäten und Relationen gleichermaßen trainiert. In den folgenden
Kapiteln soll ein Versuch unternommen werden, mithilfe von distributioneller Semantik erstellte Wortvektorrepräsentationen
für die Entitäten zu nutzen und darauf aufbauend Relationen vorherzusagen.
