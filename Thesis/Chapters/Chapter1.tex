% Chapter 1

\chapter{Einleitung} % Main chapter title

\label{Chapter1} % For referencing the chapter elsewhere, use \ref{Chapter1}

%----------------------------------------------------------------------------------------

% Define some commands to keep the formatting separated from the content
\newcommand{\keyword}[1]{\textbf{#1}}
\newcommand{\tabhead}[1]{\textbf{#1}}
\newcommand{\code}[1]{\texttt{#1}}
\newcommand{\file}[1]{\texttt{\bfseries#1}}
\newcommand{\option}[1]{\texttt{\itshape#1}}

%----------------------------------------------------------------------------------------

\begin{itquote}
"Weltwissen beschreibt das einem Individuum verfügbare allgemeine Wissen, Kenntnisse und Erfahrungen über Umwelt und Gesellschaft. [...]
Das Weltwissen ermöglicht es, neue Tatsachen einzuordnen und entsprechend zu handeln, auch wenn detaillierte Informationen fehlen. [...]

Auch in der Robotik [und KI-Forschung; Anm. des Autors] spielt Weltwissen [...] eine Rolle,
da Computer [...] nicht selbst über Weltwissen verfügen."
\flushright
\textsc{Einleitung des Artikels über Weltwissen, Wikipedia\footnote{\url{https://de.wikipedia.org/wiki/Weltwissen} (zuletzt abgerufen am 03.03.16)}}
\end{itquote}

\section{Knowledge Graph Completion}

Computer sind dem Menschen mittlerweile beim Lösen vielerlei Aufgaben überlegen. Sie rechnen schneller und genauer.
Sie können riesige Datenmengen in einem Bruchteil der Zeit verarbeiten, die ein Mensch dafür bräuchte. Die menschliche
Überlegenheit beginnt auch in Bereichen zu bröckeln, bei denen der Einsatz von Computern lange für unmöglichkeit: Menschliche
Champions scheitern nun gegen Maschinen beim Schach. Zuletzt scheiterte auch der Mensch auch beim Spiel Go gegen ein ``intelligentes'' System
\footnote{Siehe \url{http://qz.com/639952/googles-ai-won-the-game-go-by-defying-millennia-of-basic-human-instinct/} (zuletzt abgerufen am 12.07.16)}.
In anderen Bereichen jedoch hinken die Maschinen den Prognosen hinterher. Viele dieser Bereiche haben dabei eines gemeinsam:
Die Anforderung an das System, nicht nur einfache Rechenoperationen auszuführen, sondern sich ein (ggf. eingeschränktes) Bild von der umgebenden Welt zu machen,
Schlüsse zu ziehen und neues Wissen zu Erwerben und passend in den vorhandenen Informationsbestand einzuordnen. Um Probleme
für künstliche Intelligenzen lösbar zu machen, wurde bisher aber meist versucht, die Welt um das System herum zu vereinfachen:\\

\newpage
\begin{itquote}
"The [...] reason for the fundamental problems of AI in generalis that the models do
 not take the real world sufficiently into account. Much work in classical AI has been devoted to abstract, virtual
 worlds with precisely defined states and operations, quite unlike the real world."\footnote{``Problems of Traditional AI'',
 online unter \url{http://www.eucognition.org/index.php?page=2-3-problems-of-traditional-ai} (zuletzt abgerufen am 12.07.16)}
 \flushright
 \textsc{European Network for the Advancement of Artificial Cognitive Systems, Interaction
 and Robotics.}
\end{itquote}

Entwicklungen wie die ersten Fahrten fahrerloser Autos, den Jeopardy-Champion Watson und ähnliche zeigen den Fortschritt in diesem Bereich auf,
jedoch ist die Wissenschaft von einer allgemeinen Intelligenz noch weit entfernt. Ein Grund dafür ist das Problem von Computern,
dass sie im Gegensatz zum Menschen über kein Weltwissen verfügen, welches letztere sich im Laufe ihres Lebens aneignen.
Dabei lernen sie
\begin{itemize}
  \item wie Objekte in der Realwelt zueinander in Beziehung stehen
  \item welche Attribute von verschiedenen Objekten besessen werden
  \item Zusammenhänge zwischen Ereignissen zu verstehen
  \item Pläne zu schmieden und sich Stragien zurechtzulegen
  \item \ldots
\end{itemize}

Ersteres wird in der Informatik durch sog. \emph{Ontologien} (= formale Darstellung einer Beziehung zwischen Elementen einer Menge)
modelliert. Zieht man zwischen den Entitäten in der Welt nun derartige Ontologien, entsteht ein Graph, in dem die Beziehungen
der Knoten untereinander mithilfe verschiedener Arten von Kanten kodiert sieht, dem \emph{Knowledge Graph}.\\
Die Vervollständigung ebendieses ermöglicht Systemen, die langwierige menschliche Erlernung dieses Wissens zu kompensieren.
In dieser Arbeit soll deshalb ein Ansatz vorgestellt werden, der die Lösung diesen Problems einen kleinen Schritt näherbringen könnte.

\section{Ansatz}

Innerhalb der letzten Jahre haben neurale Netze in der Informatik im Allgemeinen und in der Computerlinguistik im Speziellen
eine Renaissance erlebt. Mit diesen konnten eine neue Art von Wortvektoren, auf Englisch ``word embeddings'' trainiert werden,
die semantische Information in sich kodierten. Zwar zeigen einige Untersuchung, dass sich dieser Ansatz älteren durchaus sehr
ähnlich ist und nicht unbedingt zu besseren Ergebnissen führt, der Aufruhr hat aber eine neue Welle von Forschungen im Bereich
der distributionellen Semantik ausgelöst.\\
Ein oft zitiertes Beispiel für die Ausdruckskraft dieser Vektoren ist das Entdecken von
semantischen Relationen hinter einfachen arithmetischen Operationen:

\begin{quote}
  $\vec{v}(King) - \vec{v}(Man) + \vec{v}(Woman) \approx \vec{v}(Queen)$
\end{quote}

Zwar lassen sich dadurch nicht alle Arten von Relationen aus Wortvektoren extrahieren und beschränken sich die Beispiele
bei dieser Herangehensweise auf 1:1-Relationen (1:N-, N:1- sowie N:N-Relationen lassen sich auf andere Art und Weise finden),
jedoch lassen sie schon ein gewisses Potenzial erahnen. Eine Reihe von Papern nutzt nun Methoden des maschinellen Lernens,
um mithilfe von Trainingsdaten die Differenzvektoren für bestimmte, im Voraus ausgewählte Relationen zu trainieren und die Ergebnisse
in einem Testset anzuwenden. \\
Die Idee, die in dieser Arbeit angegangen werden soll, fußt auf der Hypthese, dass das Trainieren solcher Relationen nicht
möglich wäre, wenn sich die Differenzvektoren von Wortpaaren einzelner Relationen nicht ähneln würden, also z.B.

\begin{quote}
  $\vec{v}(Berlin) - \vec{v}(Germany) \approx \vec{v}(Paris) - \vec{v}(Frankreich) \approx \vec{v}(Madrid) - \vec{v}(Spain)$
\end{quote}

Betrachtet man die Abstandsvektoren von Wortpaaren als Punkte in einem eigenen Vektorraum, so müssten theoretisch die Punkte,
die zu Ländern und deren Hauptstädten gehören, in diesem Raum nahe beieinander liegen. Dies wiederum könnte man dann insofern
ausnutzen, indem man ein Clusteringverfahren anwendet, um diese Gruppen zu identifizieren und die Elemente jeder Gruppe danach
mit der entsprechenden Relation in einen Knowledge Graph einzuordnen. Dies hätte zwei Vorteile:
\begin{enumerate}
  \item Teile des kostenintensiven Dateneinpflegens durch menschliche Hilfe fällt weg
  \item Es wird möglich abzuschätzen, welche Relationen in einem Raum von Wortvektoren tatsächlich festgehalten werden
        (darunter vielleicht auch einige, die man bei vorherigen Experimenten nicht berücksichtigt hatte)
\end{enumerate}

Die Ergebnisse, die diese Prozedur zutage fördert sowie die Fallstricke, die sie mit sich bringt, werden in den nächsten
Kapiteln beschrieben. Über jene wird nun ein kleiner Überblick gegeben.

\section{Inhalt}

In Kapitel \ref{Chapter2} dieser Arbeit sollen verwandte Arbeiten zu diesem Thema vorgestellt werden. Darauf folgt eine Beschreibung
der ein Vorbereitungsschritte, die erforderlich waren (siehe Kapitel \ref{Chapter3}) sowie eine Charakterisierung der verwendeten Ressourcen.
Dabei wird auch auf das Training verschiedener Wortvektoren eingegangen, die in Kapitel \ref{Chapter4} auf qualitative und quantitative
Art und Weise auf ihre semantische Aussagekraft evaluiert werden. \\
Das Kapitel Nummer \ref{Chapter5} handelt von Mapping-Vorgang, bei dem aus den Wortvektoren in einem neuen Vektorraum Punkte
entstehen, die jeweils einem Wortpaar entsprechen. Dabei wird auf das theoretische Verfahren genauso eingegangen wie auch
auf den benötigen Algorithmus und notwendige Einschränkungen. \\
In Kapitel \ref{Chapter6} wird der Clustering-Schritt im Bezug auf die Wahl des Algorithmus, seiner Parameter und
seiner Skalierbarkeit beschrieben. Daran schließt in Kapitel \ref{Chapter7} die Evaluation der entstehenden Cluster an,
bevor in Kapitel \ref{Chapter8} die Ergebnisse sowie die Möglichkeiten und Grenzen des Verfahrens ausdrücklich diskutiert werden. \\
Die Arbeit schließt mit einem Fazit (Kapitel \ref{Chapter9}) und einem Ausblick für auf diese Arbeit aufbauende Untersuchungen
an (Kapitel \ref{Chapter10}).
