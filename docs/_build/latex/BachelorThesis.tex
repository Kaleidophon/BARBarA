% Generated by Sphinx.
\def\sphinxdocclass{report}
\documentclass[letterpaper,10pt,english]{sphinxmanual}
\usepackage{iftex}

\ifPDFTeX
  \usepackage[utf8]{inputenc}
\fi
\ifdefined\DeclareUnicodeCharacter
  \DeclareUnicodeCharacter{00A0}{\nobreakspace}
\fi
\usepackage{cmap}
\usepackage[T1]{fontenc}
\usepackage{amsmath,amssymb,amstext}
\usepackage{babel}
\usepackage{times}
\usepackage[Bjarne]{fncychap}
\usepackage{longtable}
\usepackage{sphinx}
\usepackage{multirow}
\usepackage{eqparbox}


\addto\captionsenglish{\renewcommand{\figurename}{Fig.\@ }}
\addto\captionsenglish{\renewcommand{\tablename}{Table }}
\SetupFloatingEnvironment{literal-block}{name=Listing }

\addto\extrasenglish{\def\pageautorefname{page}}

\setcounter{tocdepth}{4}


\title{Bachelor Thesis Documentation}
\date{Jul 28, 2016}
\release{v1.0}
\author{Bachelor Thesis}
\newcommand{\sphinxlogo}{}
\renewcommand{\releasename}{Release}
\makeindex

\makeatletter
\def\PYG@reset{\let\PYG@it=\relax \let\PYG@bf=\relax%
    \let\PYG@ul=\relax \let\PYG@tc=\relax%
    \let\PYG@bc=\relax \let\PYG@ff=\relax}
\def\PYG@tok#1{\csname PYG@tok@#1\endcsname}
\def\PYG@toks#1+{\ifx\relax#1\empty\else%
    \PYG@tok{#1}\expandafter\PYG@toks\fi}
\def\PYG@do#1{\PYG@bc{\PYG@tc{\PYG@ul{%
    \PYG@it{\PYG@bf{\PYG@ff{#1}}}}}}}
\def\PYG#1#2{\PYG@reset\PYG@toks#1+\relax+\PYG@do{#2}}

\expandafter\def\csname PYG@tok@gd\endcsname{\def\PYG@tc##1{\textcolor[rgb]{0.63,0.00,0.00}{##1}}}
\expandafter\def\csname PYG@tok@gu\endcsname{\let\PYG@bf=\textbf\def\PYG@tc##1{\textcolor[rgb]{0.50,0.00,0.50}{##1}}}
\expandafter\def\csname PYG@tok@gt\endcsname{\def\PYG@tc##1{\textcolor[rgb]{0.00,0.27,0.87}{##1}}}
\expandafter\def\csname PYG@tok@gs\endcsname{\let\PYG@bf=\textbf}
\expandafter\def\csname PYG@tok@gr\endcsname{\def\PYG@tc##1{\textcolor[rgb]{1.00,0.00,0.00}{##1}}}
\expandafter\def\csname PYG@tok@cm\endcsname{\let\PYG@it=\textit\def\PYG@tc##1{\textcolor[rgb]{0.25,0.50,0.56}{##1}}}
\expandafter\def\csname PYG@tok@vg\endcsname{\def\PYG@tc##1{\textcolor[rgb]{0.73,0.38,0.84}{##1}}}
\expandafter\def\csname PYG@tok@vi\endcsname{\def\PYG@tc##1{\textcolor[rgb]{0.73,0.38,0.84}{##1}}}
\expandafter\def\csname PYG@tok@mh\endcsname{\def\PYG@tc##1{\textcolor[rgb]{0.13,0.50,0.31}{##1}}}
\expandafter\def\csname PYG@tok@cs\endcsname{\def\PYG@tc##1{\textcolor[rgb]{0.25,0.50,0.56}{##1}}\def\PYG@bc##1{\setlength{\fboxsep}{0pt}\colorbox[rgb]{1.00,0.94,0.94}{\strut ##1}}}
\expandafter\def\csname PYG@tok@ge\endcsname{\let\PYG@it=\textit}
\expandafter\def\csname PYG@tok@vc\endcsname{\def\PYG@tc##1{\textcolor[rgb]{0.73,0.38,0.84}{##1}}}
\expandafter\def\csname PYG@tok@il\endcsname{\def\PYG@tc##1{\textcolor[rgb]{0.13,0.50,0.31}{##1}}}
\expandafter\def\csname PYG@tok@go\endcsname{\def\PYG@tc##1{\textcolor[rgb]{0.20,0.20,0.20}{##1}}}
\expandafter\def\csname PYG@tok@cp\endcsname{\def\PYG@tc##1{\textcolor[rgb]{0.00,0.44,0.13}{##1}}}
\expandafter\def\csname PYG@tok@gi\endcsname{\def\PYG@tc##1{\textcolor[rgb]{0.00,0.63,0.00}{##1}}}
\expandafter\def\csname PYG@tok@gh\endcsname{\let\PYG@bf=\textbf\def\PYG@tc##1{\textcolor[rgb]{0.00,0.00,0.50}{##1}}}
\expandafter\def\csname PYG@tok@ni\endcsname{\let\PYG@bf=\textbf\def\PYG@tc##1{\textcolor[rgb]{0.84,0.33,0.22}{##1}}}
\expandafter\def\csname PYG@tok@nl\endcsname{\let\PYG@bf=\textbf\def\PYG@tc##1{\textcolor[rgb]{0.00,0.13,0.44}{##1}}}
\expandafter\def\csname PYG@tok@nn\endcsname{\let\PYG@bf=\textbf\def\PYG@tc##1{\textcolor[rgb]{0.05,0.52,0.71}{##1}}}
\expandafter\def\csname PYG@tok@no\endcsname{\def\PYG@tc##1{\textcolor[rgb]{0.38,0.68,0.84}{##1}}}
\expandafter\def\csname PYG@tok@na\endcsname{\def\PYG@tc##1{\textcolor[rgb]{0.25,0.44,0.63}{##1}}}
\expandafter\def\csname PYG@tok@nb\endcsname{\def\PYG@tc##1{\textcolor[rgb]{0.00,0.44,0.13}{##1}}}
\expandafter\def\csname PYG@tok@nc\endcsname{\let\PYG@bf=\textbf\def\PYG@tc##1{\textcolor[rgb]{0.05,0.52,0.71}{##1}}}
\expandafter\def\csname PYG@tok@nd\endcsname{\let\PYG@bf=\textbf\def\PYG@tc##1{\textcolor[rgb]{0.33,0.33,0.33}{##1}}}
\expandafter\def\csname PYG@tok@ne\endcsname{\def\PYG@tc##1{\textcolor[rgb]{0.00,0.44,0.13}{##1}}}
\expandafter\def\csname PYG@tok@nf\endcsname{\def\PYG@tc##1{\textcolor[rgb]{0.02,0.16,0.49}{##1}}}
\expandafter\def\csname PYG@tok@si\endcsname{\let\PYG@it=\textit\def\PYG@tc##1{\textcolor[rgb]{0.44,0.63,0.82}{##1}}}
\expandafter\def\csname PYG@tok@s2\endcsname{\def\PYG@tc##1{\textcolor[rgb]{0.25,0.44,0.63}{##1}}}
\expandafter\def\csname PYG@tok@nt\endcsname{\let\PYG@bf=\textbf\def\PYG@tc##1{\textcolor[rgb]{0.02,0.16,0.45}{##1}}}
\expandafter\def\csname PYG@tok@nv\endcsname{\def\PYG@tc##1{\textcolor[rgb]{0.73,0.38,0.84}{##1}}}
\expandafter\def\csname PYG@tok@s1\endcsname{\def\PYG@tc##1{\textcolor[rgb]{0.25,0.44,0.63}{##1}}}
\expandafter\def\csname PYG@tok@ch\endcsname{\let\PYG@it=\textit\def\PYG@tc##1{\textcolor[rgb]{0.25,0.50,0.56}{##1}}}
\expandafter\def\csname PYG@tok@m\endcsname{\def\PYG@tc##1{\textcolor[rgb]{0.13,0.50,0.31}{##1}}}
\expandafter\def\csname PYG@tok@gp\endcsname{\let\PYG@bf=\textbf\def\PYG@tc##1{\textcolor[rgb]{0.78,0.36,0.04}{##1}}}
\expandafter\def\csname PYG@tok@sh\endcsname{\def\PYG@tc##1{\textcolor[rgb]{0.25,0.44,0.63}{##1}}}
\expandafter\def\csname PYG@tok@ow\endcsname{\let\PYG@bf=\textbf\def\PYG@tc##1{\textcolor[rgb]{0.00,0.44,0.13}{##1}}}
\expandafter\def\csname PYG@tok@sx\endcsname{\def\PYG@tc##1{\textcolor[rgb]{0.78,0.36,0.04}{##1}}}
\expandafter\def\csname PYG@tok@bp\endcsname{\def\PYG@tc##1{\textcolor[rgb]{0.00,0.44,0.13}{##1}}}
\expandafter\def\csname PYG@tok@c1\endcsname{\let\PYG@it=\textit\def\PYG@tc##1{\textcolor[rgb]{0.25,0.50,0.56}{##1}}}
\expandafter\def\csname PYG@tok@o\endcsname{\def\PYG@tc##1{\textcolor[rgb]{0.40,0.40,0.40}{##1}}}
\expandafter\def\csname PYG@tok@kc\endcsname{\let\PYG@bf=\textbf\def\PYG@tc##1{\textcolor[rgb]{0.00,0.44,0.13}{##1}}}
\expandafter\def\csname PYG@tok@c\endcsname{\let\PYG@it=\textit\def\PYG@tc##1{\textcolor[rgb]{0.25,0.50,0.56}{##1}}}
\expandafter\def\csname PYG@tok@mf\endcsname{\def\PYG@tc##1{\textcolor[rgb]{0.13,0.50,0.31}{##1}}}
\expandafter\def\csname PYG@tok@err\endcsname{\def\PYG@bc##1{\setlength{\fboxsep}{0pt}\fcolorbox[rgb]{1.00,0.00,0.00}{1,1,1}{\strut ##1}}}
\expandafter\def\csname PYG@tok@mb\endcsname{\def\PYG@tc##1{\textcolor[rgb]{0.13,0.50,0.31}{##1}}}
\expandafter\def\csname PYG@tok@ss\endcsname{\def\PYG@tc##1{\textcolor[rgb]{0.32,0.47,0.09}{##1}}}
\expandafter\def\csname PYG@tok@sr\endcsname{\def\PYG@tc##1{\textcolor[rgb]{0.14,0.33,0.53}{##1}}}
\expandafter\def\csname PYG@tok@mo\endcsname{\def\PYG@tc##1{\textcolor[rgb]{0.13,0.50,0.31}{##1}}}
\expandafter\def\csname PYG@tok@kd\endcsname{\let\PYG@bf=\textbf\def\PYG@tc##1{\textcolor[rgb]{0.00,0.44,0.13}{##1}}}
\expandafter\def\csname PYG@tok@mi\endcsname{\def\PYG@tc##1{\textcolor[rgb]{0.13,0.50,0.31}{##1}}}
\expandafter\def\csname PYG@tok@kn\endcsname{\let\PYG@bf=\textbf\def\PYG@tc##1{\textcolor[rgb]{0.00,0.44,0.13}{##1}}}
\expandafter\def\csname PYG@tok@cpf\endcsname{\let\PYG@it=\textit\def\PYG@tc##1{\textcolor[rgb]{0.25,0.50,0.56}{##1}}}
\expandafter\def\csname PYG@tok@kr\endcsname{\let\PYG@bf=\textbf\def\PYG@tc##1{\textcolor[rgb]{0.00,0.44,0.13}{##1}}}
\expandafter\def\csname PYG@tok@s\endcsname{\def\PYG@tc##1{\textcolor[rgb]{0.25,0.44,0.63}{##1}}}
\expandafter\def\csname PYG@tok@kp\endcsname{\def\PYG@tc##1{\textcolor[rgb]{0.00,0.44,0.13}{##1}}}
\expandafter\def\csname PYG@tok@w\endcsname{\def\PYG@tc##1{\textcolor[rgb]{0.73,0.73,0.73}{##1}}}
\expandafter\def\csname PYG@tok@kt\endcsname{\def\PYG@tc##1{\textcolor[rgb]{0.56,0.13,0.00}{##1}}}
\expandafter\def\csname PYG@tok@sc\endcsname{\def\PYG@tc##1{\textcolor[rgb]{0.25,0.44,0.63}{##1}}}
\expandafter\def\csname PYG@tok@sb\endcsname{\def\PYG@tc##1{\textcolor[rgb]{0.25,0.44,0.63}{##1}}}
\expandafter\def\csname PYG@tok@k\endcsname{\let\PYG@bf=\textbf\def\PYG@tc##1{\textcolor[rgb]{0.00,0.44,0.13}{##1}}}
\expandafter\def\csname PYG@tok@se\endcsname{\let\PYG@bf=\textbf\def\PYG@tc##1{\textcolor[rgb]{0.25,0.44,0.63}{##1}}}
\expandafter\def\csname PYG@tok@sd\endcsname{\let\PYG@it=\textit\def\PYG@tc##1{\textcolor[rgb]{0.25,0.44,0.63}{##1}}}

\def\PYGZbs{\char`\\}
\def\PYGZus{\char`\_}
\def\PYGZob{\char`\{}
\def\PYGZcb{\char`\}}
\def\PYGZca{\char`\^}
\def\PYGZam{\char`\&}
\def\PYGZlt{\char`\<}
\def\PYGZgt{\char`\>}
\def\PYGZsh{\char`\#}
\def\PYGZpc{\char`\%}
\def\PYGZdl{\char`\$}
\def\PYGZhy{\char`\-}
\def\PYGZsq{\char`\'}
\def\PYGZdq{\char`\"}
\def\PYGZti{\char`\~}
% for compatibility with earlier versions
\def\PYGZat{@}
\def\PYGZlb{[}
\def\PYGZrb{]}
\makeatother

\renewcommand\PYGZsq{\textquotesingle}

\begin{document}

\maketitle
\tableofcontents
\phantomsection\label{index::doc}


Contents:


\chapter{Bachelorarbeit}
\label{index:bachelorarbeit}\label{index:welcome-to-bachelor-thesis-s-documentation}

\section{src package}
\label{src::doc}\label{src:src-package}

\subsection{Subpackages}
\label{src:subpackages}

\subsubsection{src.clustering package}
\label{src.clustering:src-clustering-package}\label{src.clustering::doc}

\paragraph{Submodules}
\label{src.clustering:submodules}

\paragraph{src.clustering.cluster\_mappings module}
\label{src.clustering:module-src.clustering.cluster_mappings}\label{src.clustering:src-clustering-cluster-mappings-module}\index{src.clustering.cluster\_mappings (module)}
Script to cluster mapping vectors created with {\hyperref[src.mapping:module\string-src.mapping.mapthreading]{\crossref{\code{src.mapping.mapthreading}}}}.
\index{aggregate\_cluster() (in module src.clustering.cluster\_mappings)}

\begin{fulllineitems}
\phantomsection\label{src.clustering:src.clustering.cluster_mappings.aggregate_cluster}\pysiglinewithargsret{\code{src.clustering.cluster\_mappings.}\bfcode{aggregate\_cluster}}{\emph{points}, \emph{labels}}{}
Arranges all clusters in a list, where a sublist with all points at index i corresponds with the
custer with label i.
\begin{quote}\begin{description}
\item[{Parameters}] \leavevmode\begin{itemize}
\item {} 
\textbf{\texttt{points}} (\emph{\texttt{list}}) -- List of datapoints

\item {} 
\textbf{\texttt{labels}} (\emph{\texttt{list}}) -- List of unique cluster labels

\end{itemize}

\item[{Returns}] \leavevmode
list of lists of datapoints belonging to the i-th cluster

\item[{Return type}] \leavevmode
list

\end{description}\end{quote}

\end{fulllineitems}

\index{cluster\_mappings() (in module src.clustering.cluster\_mappings)}

\begin{fulllineitems}
\phantomsection\label{src.clustering:src.clustering.cluster_mappings.cluster_mappings}\pysiglinewithargsret{\code{src.clustering.cluster\_mappings.}\bfcode{cluster\_mappings}}{\emph{vector\_inpath}, \emph{do\_pca=False}, \emph{target\_dim=100}, \emph{indices\_inpath=None}, \emph{epsilon=2.625}, \emph{min\_s=20}}{}
Cluster mapping vectors created with {\hyperref[src.mapping:module\string-src.mapping.mapthreading]{\crossref{\code{src.mapping.mapthreading}}}} or \code{rc.mapping.map\_vectors.py}.
Because just reading about the number of clusters and their sizes, there's an option to resolve the indices of
the vectors in the cluster to their original word pairs.
\begin{quote}\begin{description}
\item[{Parameters}] \leavevmode\begin{itemize}
\item {} 
\textbf{\texttt{vector\_inpath}} (\emph{\texttt{str}}) -- Path to vector file. File should have the following format (separated by spaces):
\textless{}index of original vector \#1\textgreater{} \textless{}index of original vector \#2\textgreater{} \textless{}Dimension 1\textgreater{} ... \textless{}Dimension n\textgreater{}

\item {} 
\textbf{\texttt{do\_pca}} (\emph{\texttt{bool}}) -- Flag to indicate whether PCA should be executed before clustering to reduce amount of

\item {} 
\textbf{\texttt{computation.}} -- 

\item {} 
\textbf{\texttt{target\_dim}} (\emph{\texttt{int}}) -- Number of dimensions vectors should be shrunk to in case PCA is performed.

\item {} 
\textbf{\texttt{indices\_inpath}} (\emph{\texttt{str}}) -- Path to file with the indices given to words. The file should have the following format:
\textless{}index of word\textgreater{} \textless{}word\textgreater{} (separated by tab)

\item {} 
\textbf{\texttt{epsilon}} (\emph{\texttt{float}}) -- Radius of circle DBSCAN uses to look for other data points.

\item {} 
\textbf{\texttt{min\_s}} (\emph{\texttt{int}}) -- Minimum number of points in radius epsilon DBSCAN needs to declare a point a core object.

\end{itemize}

\end{description}\end{quote}

\end{fulllineitems}

\index{get\_cluster\_size() (in module src.clustering.cluster\_mappings)}

\begin{fulllineitems}
\phantomsection\label{src.clustering:src.clustering.cluster_mappings.get_cluster_size}\pysiglinewithargsret{\code{src.clustering.cluster\_mappings.}\bfcode{get\_cluster\_size}}{\emph{labels}}{}
Calculate the size of every cluster found by DBSCAN.
\begin{quote}\begin{description}
\item[{Parameters}] \leavevmode
\textbf{\texttt{labels}} (\emph{\texttt{list}}) -- List of cluster IDs assigned to every data point.

\item[{Returns}] \leavevmode
Dictionary of cluster sizes with cluster id as key and cluster size as value.

\item[{Return type}] \leavevmode
defaultdict

\end{description}\end{quote}

\end{fulllineitems}

\index{init\_argparser() (in module src.clustering.cluster\_mappings)}

\begin{fulllineitems}
\phantomsection\label{src.clustering:src.clustering.cluster_mappings.init_argparser}\pysiglinewithargsret{\code{src.clustering.cluster\_mappings.}\bfcode{init\_argparser}}{}{}
Initialize all possible arguments for the argument parser.
\begin{quote}\begin{description}
\item[{Returns}] \leavevmode
ArgumentParser object with command line arguments for this script.

\item[{Return type}] \leavevmode
\code{argparse.ArgumentParser}

\end{description}\end{quote}

\end{fulllineitems}

\index{load\_indices() (in module src.clustering.cluster\_mappings)}

\begin{fulllineitems}
\phantomsection\label{src.clustering:src.clustering.cluster_mappings.load_indices}\pysiglinewithargsret{\code{src.clustering.cluster\_mappings.}\bfcode{load\_indices}}{\emph{indices\_inpath}}{}
Load word indices from a file. The file should have the following format: \textless{}index of word\textgreater{}       \textless{}word\textgreater{} (separated by
tab)
\begin{quote}\begin{description}
\item[{Parameters}] \leavevmode
\textbf{\texttt{indices\_inpath}} (\emph{\texttt{str}}) -- Path to index file.

\end{description}\end{quote}

\end{fulllineitems}

\index{load\_mappings\_from\_model() (in module src.clustering.cluster\_mappings)}

\begin{fulllineitems}
\phantomsection\label{src.clustering:src.clustering.cluster_mappings.load_mappings_from_model}\pysiglinewithargsret{\code{src.clustering.cluster\_mappings.}\bfcode{load\_mappings\_from\_model}}{\emph{mapping\_inpath}}{}
Load mapping vectors from file.
\begin{quote}\begin{description}
\item[{Parameters}] \leavevmode
\textbf{\texttt{mapping\_inpath}} -- Path mapping vector file.

\item[{Returns}] \leavevmode
A tuple of a list of word index pairs and a dictionary (defaultdict) with index pair tuple as key
and mapping vector (as numpy.array) as value.

\item[{Return type}] \leavevmode
tuple

\end{description}\end{quote}

\end{fulllineitems}

\index{main() (in module src.clustering.cluster\_mappings)}

\begin{fulllineitems}
\phantomsection\label{src.clustering:src.clustering.cluster_mappings.main}\pysiglinewithargsret{\code{src.clustering.cluster\_mappings.}\bfcode{main}}{}{}
This is the main function. It uses the parsed command line arguments to pick the right function to execute.

\end{fulllineitems}

\index{resolve\_indices() (in module src.clustering.cluster\_mappings)}

\begin{fulllineitems}
\phantomsection\label{src.clustering:src.clustering.cluster_mappings.resolve_indices}\pysiglinewithargsret{\code{src.clustering.cluster\_mappings.}\bfcode{resolve\_indices}}{\emph{points}, \emph{labels}, \emph{indices\_inpath}, \emph{model}}{}
\end{fulllineitems}

\index{train\_clustering\_parameters() (in module src.clustering.cluster\_mappings)}

\begin{fulllineitems}
\phantomsection\label{src.clustering:src.clustering.cluster_mappings.train_clustering_parameters}\pysiglinewithargsret{\code{src.clustering.cluster\_mappings.}\bfcode{train\_clustering\_parameters}}{\emph{vector\_inpath}}{}
Functions that tries to figure out the optimal clustering parameters in regard to DBSCAN's epsilon,
min\_samples and p.
\begin{quote}\begin{description}
\item[{Parameters}] \leavevmode
\textbf{\texttt{vector\_inpath}} (\emph{\texttt{str}}) -- Path to vector file. File has to have the following format (separated by spaces):
\textless{}index of original vector \#1\textgreater{} \textless{}index of original vector \#2\textgreater{} \textless{}Dimension 1\textgreater{} ... \textless{}Dimension n\textgreater{}

\end{description}\end{quote}

\end{fulllineitems}



\paragraph{Module contents}
\label{src.clustering:module-contents}\label{src.clustering:module-src.clustering}\index{src.clustering (module)}

\subsubsection{src.eval package}
\label{src.eval::doc}\label{src.eval:src-eval-package}

\paragraph{Submodules}
\label{src.eval:submodules}

\paragraph{src.eval.analogy module}
\label{src.eval:module-src.eval.analogy}\label{src.eval:src-eval-analogy-module}\index{src.eval.analogy (module)}
Module to evaluate word embeddings by the means of analogies like ``W is to X like Y is to Z''. Usually,
the system uses the word embeddings of word W, X, Y and tries to find the vector of word Z that is most similar to
X and Y and most dissimilar to W.
Therefore, the \href{https://transacl.org/ojs/index.php/tacl/article/viewFile/570/124}{CosMul method (Levy et al., 2015)}
is used.

The whole module is used in \code{src.eval.eval\_vectors.py}.
\index{analogy\_eval() (in module src.eval.analogy)}

\begin{fulllineitems}
\phantomsection\label{src.eval:src.eval.analogy.analogy_eval}\pysiglinewithargsret{\code{src.eval.analogy.}\bfcode{analogy\_eval}}{\emph{vector\_inpath}, \emph{analogy\_path}, \emph{per\_section=False}}{}
Perform analogy evaluation. Usually, the system uses the word embeddings of word W, X, Y and tries to find the
vector of word Z that is most similar to
X and Y and most dissimilar to W for an analogy like ``W is to X like Y is to Z.''
Therefore, the CosMul method (Levy et al., 2015) is used.
\begin{quote}\begin{description}
\item[{Parameters}] \leavevmode\begin{itemize}
\item {} 
\textbf{\texttt{vector\_inpath}} (\emph{\texttt{str}}) -- Path to \titleref{word2vec} vector file.

\item {} 
\textbf{\texttt{analogy\_path}} (\emph{\texttt{str}}) -- Path to analogy file.

\item {} 
\textbf{\texttt{per\_section}} (\emph{\texttt{bool}}) -- Flag to indicate whether analogies test should be conducted section-wise or just all in
one run.

\end{itemize}

\end{description}\end{quote}

\end{fulllineitems}

\index{read\_analogies() (in module src.eval.analogy)}

\begin{fulllineitems}
\phantomsection\label{src.eval:src.eval.analogy.read_analogies}\pysiglinewithargsret{\code{src.eval.analogy.}\bfcode{read\_analogies}}{\emph{analogy\_path}, \emph{per\_section=False}}{}
Reads a file with analogies.
\begin{quote}\begin{description}
\item[{Parameters}] \leavevmode\begin{itemize}
\item {} 
\textbf{\texttt{analogy\_path}} (\emph{\texttt{str}}) -- Path to analogy file.

\item {} 
\textbf{\texttt{per\_section}} (\emph{\texttt{bool}}) -- Flag to indicate whether analogies test should be conducted section-wise or just all in
one run. In this function, the section will be put into a data structure accordingly.

\end{itemize}

\item[{Returns}] \leavevmode
Dictionary with section header as key, list of analogy as 4-tuples as value.

\item[{Return type}] \leavevmode
dict

\end{description}\end{quote}

\end{fulllineitems}



\paragraph{src.eval.eval\_vectors module}
\label{src.eval:src-eval-eval-vectors-module}\label{src.eval:module-src.eval.eval_vectors}\index{src.eval.eval\_vectors (module)}\begin{description}
\item[{Main module used to evaluate word embeddings. It offers the following options:}] \leavevmode
\begin{DUlineblock}{0em}
\item[] 1.) Analogy: The system tries to complete an analogy like ``W is to X like Y is to...?'' The percentage
of correct answers is measured.
\item[] 2.) Word similarity: The system assign word pairs a similarity score based on the cosine similarity of their
word embeddings. Then, to correlation between those and human ratings is measured with Pearson's rho.
\item[] 3.) Nearest neighbors: Find the nearest neighbors for a list of words based on their word embeddings. Good for
a first look on the data, but not quantifiable.
\item[] 4.) Visualize: Plot word embeddings in 2D or 3D. Fancy plots. Yay!
\end{DUlineblock}

\end{description}
\index{find\_nearest\_neighbors() (in module src.eval.eval\_vectors)}

\begin{fulllineitems}
\phantomsection\label{src.eval:src.eval.eval_vectors.find_nearest_neighbors}\pysiglinewithargsret{\code{src.eval.eval\_vectors.}\bfcode{find\_nearest\_neighbors}}{\emph{vector\_inpath}, \emph{max\_n}, \emph{wordlist}}{}
Find the nearest neighbors for a list of words based on their word embeddings.
\begin{quote}\begin{description}
\item[{Parameters}] \leavevmode\begin{itemize}
\item {} 
\textbf{\texttt{vector\_inpath}} (\emph{\texttt{str}}) -- Path to vector file. File has to have the following format (separated by spaces):
\textless{}index of original vector \#1\textgreater{} \textless{}index of original vector \#2\textgreater{} \textless{}Dimension 1\textgreater{} ... \textless{}Dimension n\textgreater{}

\item {} 
\textbf{\texttt{max\_n}} (\emph{\texttt{int}}) -- Number of nearest neighbors that should be determined.

\item {} 
\textbf{\texttt{wordlist}} (\emph{\texttt{list}}) -- List of words nearest neighbors should be found for.

\end{itemize}

\end{description}\end{quote}

\end{fulllineitems}

\index{init\_argparser() (in module src.eval.eval\_vectors)}

\begin{fulllineitems}
\phantomsection\label{src.eval:src.eval.eval_vectors.init_argparser}\pysiglinewithargsret{\code{src.eval.eval\_vectors.}\bfcode{init\_argparser}}{}{}
Initialize all possible arguments for the argument parser.
\begin{quote}\begin{description}
\item[{Returns}] \leavevmode
ArgumentParser object with command line arguments for this script.

\item[{Return type}] \leavevmode
\code{argparse.ArgumentParser}

\end{description}\end{quote}

\end{fulllineitems}

\index{main() (in module src.eval.eval\_vectors)}

\begin{fulllineitems}
\phantomsection\label{src.eval:src.eval.eval_vectors.main}\pysiglinewithargsret{\code{src.eval.eval\_vectors.}\bfcode{main}}{}{}
This is the main function. It uses the parsed command line arguments, especially \titleref{--mode}, to pick the right
function to execute.

\end{fulllineitems}

\index{plot() (in module src.eval.eval\_vectors)}

\begin{fulllineitems}
\phantomsection\label{src.eval:src.eval.eval_vectors.plot}\pysiglinewithargsret{\code{src.eval.eval\_vectors.}\bfcode{plot}}{\emph{vector\_inpath}, \emph{max\_n}, \emph{target\_dim}, \emph{show\_plot=False}, \emph{display\_names=False}}{}
Plot word embeddings in 2D or 3D. As a heuristic, word will only be plotted after the 50th most frequent words to
avoid plotting boring stop words.
\begin{quote}\begin{description}
\item[{Parameters}] \leavevmode\begin{itemize}
\item {} 
\textbf{\texttt{vector\_inpath}} (\emph{\texttt{str}}) -- Path to vector file. File has to have the following format (separated by spaces):
\textless{}index of original vector \#1\textgreater{} \textless{}index of original vector \#2\textgreater{} \textless{}Dimension 1\textgreater{} ... \textless{}Dimension n\textgreater{}

\item {} 
\textbf{\texttt{max\_n}} (\emph{\texttt{int}}) -- Maximum number of vectors to be plotted.

\item {} 
\textbf{\texttt{show\_plot}} (\emph{\texttt{bool}}) -- Flag to indicate whether a window with the (interactive) plot should pop up after executing
the script.

\item {} 
\textbf{\texttt{display\_names}} (\emph{\texttt{bool}}) -- Flag to indicate whether the words should acutally be shown next to the data point in
the plot. Can get very messy with higher \titleref{max\_n}.

\end{itemize}

\end{description}\end{quote}

\end{fulllineitems}



\paragraph{src.eval.word\_similarity module}
\label{src.eval:src-eval-word-similarity-module}\label{src.eval:module-src.eval.word_similarity}\index{src.eval.word\_similarity (module)}
Module used to conduct the word similarity evaluation. The system assign word pairs a similarity score based on
the cosine similarity of their word embeddings. Then, to correlation between those and human ratings is measured with
Pearson's rho.

The whole module is used in \code{src.eval.eval\_vectors.py}.
\index{evaluate\_wordpair\_sims() (in module src.eval.word\_similarity)}

\begin{fulllineitems}
\phantomsection\label{src.eval:src.eval.word_similarity.evaluate_wordpair_sims}\pysiglinewithargsret{\code{src.eval.word\_similarity.}\bfcode{evaluate\_wordpair\_sims}}{\emph{x}, \emph{y}, \emph{number\_of\_pairs}}{}
Evaluate results of the similarity score assignments, i.e. calculate pearson's rho and its significance.
\begin{quote}\begin{description}
\item[{Parameters}] \leavevmode\begin{itemize}
\item {} 
\textbf{\texttt{x}} (\emph{\texttt{list}}) -- List of similarity scores assigned by humans.

\item {} 
\textbf{\texttt{y}} (\emph{\texttt{list}}) -- List of similarity scores assigned by the system.

\item {} 
\textbf{\texttt{number\_of\_pairs}} (\emph{\texttt{int}}) -- Number of word pairs evaluated.

\end{itemize}

\item[{Returns}] \leavevmode
\textbf{rho} -- Pearson's correlation coefficient.
t (float): Student's t value.
z (float): z value.

\item[{Return type}] \leavevmode
float

\end{description}\end{quote}

\end{fulllineitems}

\index{read\_wordpairs() (in module src.eval.word\_similarity)}

\begin{fulllineitems}
\phantomsection\label{src.eval:src.eval.word_similarity.read_wordpairs}\pysiglinewithargsret{\code{src.eval.word\_similarity.}\bfcode{read\_wordpairs}}{\emph{wordpair\_path}, \emph{format='google'}}{}
Read wordpair file with wordpairs and their similarity scores assigned by humans.
\begin{quote}\begin{description}
\item[{Parameters}] \leavevmode\begin{itemize}
\item {} 
\textbf{\texttt{wordpair\_path}} (\emph{\texttt{str}}) -- Path to word pair file.

\item {} 
\textbf{\texttt{format}} (\emph{\texttt{str}}) -- Format of wor pair file \{google\textbar{}semrel\}

\end{itemize}

\item[{Returns}] \leavevmode
Tuple of a list of word pairs and a list of similarity scores for those same pair assigned by humans.

\item[{Return type}] \leavevmode
tuple

\end{description}\end{quote}

\end{fulllineitems}

\index{remove\_unknowns() (in module src.eval.word\_similarity)}

\begin{fulllineitems}
\phantomsection\label{src.eval:src.eval.word_similarity.remove_unknowns}\pysiglinewithargsret{\code{src.eval.word\_similarity.}\bfcode{remove\_unknowns}}{\emph{x}, \emph{y}}{}
Remove word pairs from the results where one or two word embedding weren't found.
\begin{quote}\begin{description}
\item[{Parameters}] \leavevmode\begin{itemize}
\item {} 
\textbf{\texttt{x}} (\emph{\texttt{list}}) -- List of similarity scores assigned by humans.

\item {} 
\textbf{\texttt{y}} (\emph{\texttt{list}}) -- List of similarity scores assigned by the system.

\end{itemize}

\item[{Returns}] \leavevmode
\textbf{x} -- Purged list of similarity scores assigned by humans.
y (list): Purged list of similarity scores assigned by the system.

\item[{Return type}] \leavevmode
list

\end{description}\end{quote}

\end{fulllineitems}

\index{word\_sim\_eval() (in module src.eval.word\_similarity)}

\begin{fulllineitems}
\phantomsection\label{src.eval:src.eval.word_similarity.word_sim_eval}\pysiglinewithargsret{\code{src.eval.word\_similarity.}\bfcode{word\_sim\_eval}}{\emph{vector\_inpath}, \emph{wordpair\_path}, \emph{format='google'}}{}
Function that let's the system assign word pairs a similarity score based on the cosine similarity of their word
embeddings. Then, to correlation between those and human ratings is measured with Pearson's rho.
\begin{quote}\begin{description}
\item[{Parameters}] \leavevmode\begin{itemize}
\item {} 
\textbf{\texttt{vector\_inpath}} (\emph{\texttt{str}}) -- Path to vector file. File has to have the following format (separated by spaces):
\textless{}index of original vector \#1\textgreater{} \textless{}index of original vector \#2\textgreater{} \textless{}Dimension 1\textgreater{} ... \textless{}Dimension n\textgreater{}

\item {} 
\textbf{\texttt{wordpair\_path}} (\emph{\texttt{str}}) -- Path to word pair file.

\item {} 
\textbf{\texttt{format}} (\emph{\texttt{str}}) -- Format of word pair file \{google\textbar{}semrel\}

\end{itemize}

\end{description}\end{quote}

\end{fulllineitems}



\paragraph{Module contents}
\label{src.eval:module-src.eval}\label{src.eval:module-contents}\index{src.eval (module)}

\subsubsection{src.mapping package}
\label{src.mapping::doc}\label{src.mapping:src-mapping-package}

\paragraph{Submodules}
\label{src.mapping:submodules}

\paragraph{src.mapping.mapthreading module}
\label{src.mapping:src-mapping-mapthreading-module}\label{src.mapping:module-src.mapping.mapthreading}\index{src.mapping.mapthreading (module)}
Module used to map a pair of vectors into a new combined vector space. Those mappings will be created by multiple
threads in a master-slave-pattern. To do so, the user can choose between different vector operations as offset, cosine
similarity, euclidean distance and many more.

\begin{notice}{warning}{Warning:}
Because of \(\Omega=\frac{n(n-1)}{2}\), it is recommended to use the co-occurrence constraint
\(\Lambda\), which limits the calculations to word embedding pairs which words occurred together in a corpus in
at least \emph{n} sentences (but it will still take quite a while).
\end{notice}
\index{MappingMasterThread (class in src.mapping.mapthreading)}

\begin{fulllineitems}
\phantomsection\label{src.mapping:src.mapping.mapthreading.MappingMasterThread}\pysiglinewithargsret{\strong{class }\code{src.mapping.mapthreading.}\bfcode{MappingMasterThread}}{\emph{n}, \emph{vector\_inpath}, \emph{vector\_outpath}, \emph{features}, \emph{lambda\_}, \emph{ids\_inpath}, \emph{indices\_inpath}}{}
Bases: \code{threading.Thread}

Master thread class. The master thread loads all necessary data into suitable data structures and distributes
them among all worker threads.
\index{prepare() (src.mapping.mapthreading.MappingMasterThread method)}

\begin{fulllineitems}
\phantomsection\label{src.mapping:src.mapping.mapthreading.MappingMasterThread.prepare}\pysiglinewithargsret{\bfcode{prepare}}{}{}
Loads The master thread loads all necessary data into suitable data structures. To be more specific,
word embeddings, sentence IDs and word indices are processed.

\end{fulllineitems}

\index{read\_ids\_file() (src.mapping.mapthreading.MappingMasterThread method)}

\begin{fulllineitems}
\phantomsection\label{src.mapping:src.mapping.mapthreading.MappingMasterThread.read_ids_file}\pysiglinewithargsret{\bfcode{read\_ids\_file}}{\emph{ids\_inpath}}{}
Read the sentence ID file.
\begin{quote}\begin{description}
\item[{Parameters}] \leavevmode
\textbf{\texttt{ids\_inpath}} (\emph{\texttt{str}}) -- 
Path to sentence IDs file. The file should be in the following \titleref{YAML}-format:
- \textless{}word\textgreater{}:
\begin{quote}
\begin{itemize}
\item {} 
\textless{}sentence id\textgreater{}

\item {} 
\textless{}sentence id\textgreater{}

\end{itemize}

...
\end{quote}


\item[{Returns}] \leavevmode
Dictionary with words as keys and the IDs of the sentences they occur in in a set as value.

\item[{Return type}] \leavevmode
defaultdict

\end{description}\end{quote}

\end{fulllineitems}

\index{start\_threads() (src.mapping.mapthreading.MappingMasterThread method)}

\begin{fulllineitems}
\phantomsection\label{src.mapping:src.mapping.mapthreading.MappingMasterThread.start_threads}\pysiglinewithargsret{\bfcode{start\_threads}}{}{}
Starts all the threads (and ends them if they're all finished).

\end{fulllineitems}


\end{fulllineitems}

\index{MappingWorkerThread (class in src.mapping.mapthreading)}

\begin{fulllineitems}
\phantomsection\label{src.mapping:src.mapping.mapthreading.MappingWorkerThread}\pysiglinewithargsret{\strong{class }\code{src.mapping.mapthreading.}\bfcode{MappingWorkerThread}}{\emph{worker\_id}, \emph{vector\_dict}, \emph{vector\_queue}, \emph{vector\_outpath}, \emph{features}, \emph{occurrences}, \emph{indices}, \emph{lambda\_}}{}
Bases: \code{threading.Thread}

Worker thread class. The worker threads do all the dirty work after they receive all necessary data from the master
thread an try to calculate every possible combinations of two word embeddings in a dataset.

All the word embeddings will be stores in a dictionary ({\hyperref[src.mapping:src.mapping.mapthreading.VectorDict]{\crossref{\code{VectorDict}}}} as well as a \code{Queue}).
An idle thread picks a new vector from the queue and then starts to iterate over all the vectors in the
{\hyperref[src.mapping:src.mapping.mapthreading.VectorDict]{\crossref{\code{VectorDict}}}} (this way, the queue gets shorter over time while the size of the dictionary stays fixed).

Before it starts calculations, it checks a) if the co-occurrence constraint is satisfied and b) if this combination
of word embeddings has already been processed.
\index{cosine\_similarity() (src.mapping.mapthreading.MappingWorkerThread method)}

\begin{fulllineitems}
\phantomsection\label{src.mapping:src.mapping.mapthreading.MappingWorkerThread.cosine_similarity}\pysiglinewithargsret{\bfcode{cosine\_similarity}}{\emph{v1}, \emph{v2}}{}
Calculates the cosine similarity (\(cos(\vec{v}_1, \vec{v}_2) \in [-1,-1]\)) between two vectors.
\begin{quote}\begin{description}
\item[{Parameters}] \leavevmode\begin{itemize}
\item {} 
\textbf{\texttt{v1}} (\emph{\texttt{numpy.array}}) -- First vector

\item {} 
\textbf{\texttt{v2}} (\emph{\texttt{numpy.array}}) -- Second vector

\end{itemize}

\item[{Returns}] \leavevmode
Cosine similarity between the two vectors.

\item[{Return type}] \leavevmode
float

\end{description}\end{quote}

\end{fulllineitems}

\index{distance() (src.mapping.mapthreading.MappingWorkerThread method)}

\begin{fulllineitems}
\phantomsection\label{src.mapping:src.mapping.mapthreading.MappingWorkerThread.distance}\pysiglinewithargsret{\bfcode{distance}}{\emph{v1}, \emph{v2}}{}
Return the vector offset of two vectors:
\begin{quote}\begin{description}
\item[{Parameters}] \leavevmode\begin{itemize}
\item {} 
\textbf{\texttt{v1}} (\emph{\texttt{numpy.array}}) -- First vector

\item {} 
\textbf{\texttt{v2}} (\emph{\texttt{numpy.array}}) -- Second vector

\end{itemize}

\end{description}\end{quote}
\begin{description}
\item[{Returns}] \leavevmode
numpy.array: Vector offset.

\end{description}

\end{fulllineitems}

\index{euclidean\_distance1() (src.mapping.mapthreading.MappingWorkerThread method)}

\begin{fulllineitems}
\phantomsection\label{src.mapping:src.mapping.mapthreading.MappingWorkerThread.euclidean_distance1}\pysiglinewithargsret{\bfcode{euclidean\_distance1}}{\emph{v1}, \emph{v2}}{}
Return the euclidean distance between two vectors.
\begin{equation*}
\begin{split}eucl(\vec{a}, \vec{b}) = \sqrt{\sum_{i=1}^n (\vec{b}_i - \vec{a}_i)^2}\end{split}
\end{equation*}\begin{quote}\begin{description}
\item[{Parameters}] \leavevmode\begin{itemize}
\item {} 
\textbf{\texttt{v1}} (\emph{\texttt{numpy.array}}) -- First vector

\item {} 
\textbf{\texttt{v2}} (\emph{\texttt{numpy.array}}) -- Second vector

\end{itemize}

\item[{Returns}] \leavevmode
Euclidean distance between the two vectors.

\item[{Return type}] \leavevmode
float

\end{description}\end{quote}

\end{fulllineitems}

\index{euclidean\_distance2() (src.mapping.mapthreading.MappingWorkerThread method)}

\begin{fulllineitems}
\phantomsection\label{src.mapping:src.mapping.mapthreading.MappingWorkerThread.euclidean_distance2}\pysiglinewithargsret{\bfcode{euclidean\_distance2}}{\emph{v1}, \emph{v2}}{}
Returns the squared euclidean distance between two vectors.
\begin{equation*}
\begin{split}eucl2(\vec{a}, \vec{b}) = \sum_{i=1}^n (\vec{b}_i - \vec{a}_i)^2\end{split}
\end{equation*}\begin{quote}\begin{description}
\item[{Parameters}] \leavevmode\begin{itemize}
\item {} 
\textbf{\texttt{v1}} (\emph{\texttt{numpy.array}}) -- First vector

\item {} 
\textbf{\texttt{v2}} (\emph{\texttt{numpy.array}}) -- Second vector

\end{itemize}

\item[{Returns}] \leavevmode
Squared euclidean distance between the two vectors.

\item[{Return type}] \leavevmode
float

\end{description}\end{quote}

\end{fulllineitems}

\index{hash\_indices() (src.mapping.mapthreading.MappingWorkerThread method)}

\begin{fulllineitems}
\phantomsection\label{src.mapping:src.mapping.mapthreading.MappingWorkerThread.hash_indices}\pysiglinewithargsret{\bfcode{hash\_indices}}{\emph{i1}, \emph{i2}}{}
Combines two vector indices (the indices of the words' embeddings used in vector operations) into a hash
s.t. threads can do an easy lookup if a mapping vector has already been calculated.
To guarantee this, \(h(i_1, i_2) = h(i_2, i_1)\) has to be the case.
\begin{quote}\begin{description}
\item[{Parameters}] \leavevmode\begin{itemize}
\item {} 
\textbf{\texttt{i1}} (\emph{\texttt{int}}) -- Index of first word's embedding

\item {} 
\textbf{\texttt{i2}} (\emph{\texttt{int}}) -- Index of second word's embedding

\end{itemize}

\item[{Returns}] \leavevmode
Unique hash for index pair.

\item[{Return type}] \leavevmode
int

\end{description}\end{quote}

\end{fulllineitems}

\index{manhattan\_distance() (src.mapping.mapthreading.MappingWorkerThread method)}

\begin{fulllineitems}
\phantomsection\label{src.mapping:src.mapping.mapthreading.MappingWorkerThread.manhattan_distance}\pysiglinewithargsret{\bfcode{manhattan\_distance}}{\emph{v1}, \emph{v2}}{}
Returns the manhattan distance between two vectors.
\begin{equation*}
\begin{split}manhattan(\vec{a}, \vec{b}) = \sum_{i=1}^n |\vec{b}_i - \vec{a}_i |\end{split}
\end{equation*}\begin{quote}\begin{description}
\item[{Parameters}] \leavevmode\begin{itemize}
\item {} 
\textbf{\texttt{v1}} (\emph{\texttt{numpy.array}}) -- First vector

\item {} 
\textbf{\texttt{v2}} (\emph{\texttt{numpy.array}}) -- Second vector

\end{itemize}

\item[{Returns}] \leavevmode
Manhattan distance between the two vectors.

\item[{Return type}] \leavevmode
float

\end{description}\end{quote}

\end{fulllineitems}

\index{run() (src.mapping.mapthreading.MappingWorkerThread method)}

\begin{fulllineitems}
\phantomsection\label{src.mapping:src.mapping.mapthreading.MappingWorkerThread.run}\pysiglinewithargsret{\bfcode{run}}{}{}
Starts a worker thread.

\end{fulllineitems}

\index{soft\_cosine\_similarity() (src.mapping.mapthreading.MappingWorkerThread method)}

\begin{fulllineitems}
\phantomsection\label{src.mapping:src.mapping.mapthreading.MappingWorkerThread.soft_cosine_similarity}\pysiglinewithargsret{\bfcode{soft\_cosine\_similarity}}{\emph{v1}, \emph{v2}}{}
Calculates the soft cosine similarity between two vectors.
\begin{align*}\!\begin{aligned}
S = \begin{bmatrix}
        eucl(\vec{a}_1, \vec{b}_1) & \ldots & eucl(\vec{a}_1, \vec{b}_n) \\
        \vdots & \ddots & \vdots \\
        eucl(\vec{a}_n, \vec{b}_1) & \ldots & eucl(\vec{a}_n, \vec{b}_n) \\
\end{bmatrix}\\
softcos(\vec{a}, \vec{b}) = \frac{\sum_{i,j}^N S_{ij}\vec{a}_i\vec{b}_j}{\sqrt{\sum_{i,
j}^N S_{ij}\vec{a}_i\vec{a}_j}\sqrt{\sum_{i,j}^N S_{ij}\vec{b}_i\vec{b}_j}}\\
\end{aligned}\end{align*}
(It considers the similarity between pairs of features.)
\begin{quote}\begin{description}
\item[{Parameters}] \leavevmode\begin{itemize}
\item {} 
\textbf{\texttt{v1}} (\emph{\texttt{numpy.array}}) -- First vector

\item {} 
\textbf{\texttt{v2}} (\emph{\texttt{numpy.array}}) -- Second vector

\end{itemize}

\item[{Returns}] \leavevmode
Soft cosine similarity between the two vectors.

\item[{Return type}] \leavevmode
float

\end{description}\end{quote}

\end{fulllineitems}


\end{fulllineitems}

\index{VectorDict (class in src.mapping.mapthreading)}

\begin{fulllineitems}
\phantomsection\label{src.mapping:src.mapping.mapthreading.VectorDict}\pysigline{\strong{class }\code{src.mapping.mapthreading.}\bfcode{VectorDict}}
Bases: \code{object}
\begin{description}
\item[{VectorDict class that serves two functions:}] \leavevmode
\begin{DUlineblock}{0em}
\item[] 1.) Storing word embeddings so they don't allocate memory for every worker thread
\item[] 2.) Providing a set, where are processed vector pairs are stored so no redundant computations are made.
\end{DUlineblock}

\end{description}

Locks are used for synchronization purposes.
\index{add\_skippable() (src.mapping.mapthreading.VectorDict method)}

\begin{fulllineitems}
\phantomsection\label{src.mapping:src.mapping.mapthreading.VectorDict.add_skippable}\pysiglinewithargsret{\bfcode{add\_skippable}}{\emph{index\_hash}}{}
Add the hash of an index pair to a set of already processed vector pairs.
\begin{quote}\begin{description}
\item[{Parameters}] \leavevmode
\textbf{\texttt{index\_hash}} (\emph{\texttt{int}}) -- Hash value of index pair. Produced with \code{hash\_indices()}.

\end{description}\end{quote}

\end{fulllineitems}

\index{add\_vector() (src.mapping.mapthreading.VectorDict method)}

\begin{fulllineitems}
\phantomsection\label{src.mapping:src.mapping.mapthreading.VectorDict.add_vector}\pysiglinewithargsret{\bfcode{add\_vector}}{\emph{index}, \emph{vector}}{}
Add a new word embedding.
\begin{quote}\begin{description}
\item[{Parameters}] \leavevmode\begin{itemize}
\item {} 
\textbf{\texttt{index}} (\emph{\texttt{int}}) -- Index of the word the embedding belongs to.

\item {} 
\textbf{\texttt{vector}} (\emph{\texttt{numpy.array}}) -- Word embedding corresponding to given index.

\end{itemize}

\end{description}\end{quote}

\end{fulllineitems}

\index{get\_keys() (src.mapping.mapthreading.VectorDict method)}

\begin{fulllineitems}
\phantomsection\label{src.mapping:src.mapping.mapthreading.VectorDict.get_keys}\pysiglinewithargsret{\bfcode{get\_keys}}{}{}
Get all the keys (word embedding IDs) of this dictionary.
\begin{quote}\begin{description}
\item[{Returns}] \leavevmode
List of word embedding IDs.

\item[{Return type}] \leavevmode
list

\end{description}\end{quote}

\end{fulllineitems}

\index{get\_vector() (src.mapping.mapthreading.VectorDict method)}

\begin{fulllineitems}
\phantomsection\label{src.mapping:src.mapping.mapthreading.VectorDict.get_vector}\pysiglinewithargsret{\bfcode{get\_vector}}{\emph{index}}{}
Get a word embedding given its word's index.
\begin{quote}\begin{description}
\item[{Parameters}] \leavevmode
\textbf{\texttt{index}} (\emph{\texttt{int}}) -- Index of the word the embedding belongs to.

\item[{Returns}] \leavevmode
Word embedding corresponding to given index.

\item[{Return type}] \leavevmode
numpy.array

\end{description}\end{quote}

\end{fulllineitems}

\index{skippable() (src.mapping.mapthreading.VectorDict method)}

\begin{fulllineitems}
\phantomsection\label{src.mapping:src.mapping.mapthreading.VectorDict.skippable}\pysiglinewithargsret{\bfcode{skippable}}{\emph{index\_hash}}{}
Checks whether a pair of vectors has already been processed.
\begin{quote}\begin{description}
\item[{Parameters}] \leavevmode
\textbf{\texttt{index\_hash}} (\emph{\texttt{int}}) -- Hash value of index pair. Produced with \code{hash\_indices()}.

\item[{Returns}] \leavevmode
Whether a pair of vectors has already been processed.

\item[{Return type}] \leavevmode
bool

\end{description}\end{quote}

\end{fulllineitems}


\end{fulllineitems}

\index{alt() (in module src.mapping.mapthreading)}

\begin{fulllineitems}
\phantomsection\label{src.mapping:src.mapping.mapthreading.alt}\pysiglinewithargsret{\code{src.mapping.mapthreading.}\bfcode{alt}}{\emph{func}}{}
Prepends the local time to the output of a function.
\begin{quote}\begin{description}
\item[{Parameters}] \leavevmode
\textbf{\texttt{func}} (\emph{\texttt{function}}) -- Function the local time should be prepended to.

\end{description}\end{quote}

\end{fulllineitems}

\index{init\_argparse() (in module src.mapping.mapthreading)}

\begin{fulllineitems}
\phantomsection\label{src.mapping:src.mapping.mapthreading.init_argparse}\pysiglinewithargsret{\code{src.mapping.mapthreading.}\bfcode{init\_argparse}}{}{}
Initialize all possible arguments for the argument parser.
\begin{quote}\begin{description}
\item[{Returns}] \leavevmode
ArgumentParser object with command line arguments for this script.

\item[{Return type}] \leavevmode
\code{argparse.ArgumentParser}

\end{description}\end{quote}

\end{fulllineitems}

\index{main() (in module src.mapping.mapthreading)}

\begin{fulllineitems}
\phantomsection\label{src.mapping:src.mapping.mapthreading.main}\pysiglinewithargsret{\code{src.mapping.mapthreading.}\bfcode{main}}{}{}
Main function that initializes the master thread with command line arguments and starts it.

\end{fulllineitems}



\paragraph{Module contents}
\label{src.mapping:module-src.mapping}\label{src.mapping:module-contents}\index{src.mapping (module)}

\subsubsection{src.misc package}
\label{src.misc:src-misc-package}\label{src.misc::doc}

\paragraph{Submodules}
\label{src.misc:submodules}

\paragraph{src.misc.decorators module}
\label{src.misc:module-src.misc.decorators}\label{src.misc:src-misc-decorators-module}\index{src.misc.decorators (module)}\index{alt() (in module src.misc.decorators)}

\begin{fulllineitems}
\phantomsection\label{src.misc:src.misc.decorators.alt}\pysiglinewithargsret{\code{src.misc.decorators.}\bfcode{alt}}{\emph{func}}{}
\end{fulllineitems}

\index{log\_time() (in module src.misc.decorators)}

\begin{fulllineitems}
\phantomsection\label{src.misc:src.misc.decorators.log_time}\pysiglinewithargsret{\code{src.misc.decorators.}\bfcode{log\_time}}{\emph{logpath='log.txt'}, \emph{interval=5}}{}
\end{fulllineitems}

\index{log\_time\_mp() (in module src.misc.decorators)}

\begin{fulllineitems}
\phantomsection\label{src.misc:src.misc.decorators.log_time_mp}\pysiglinewithargsret{\code{src.misc.decorators.}\bfcode{log\_time\_mp}}{\emph{logpath='log.txt'}, \emph{interval=5}}{}
\end{fulllineitems}



\paragraph{src.misc.helpers module}
\label{src.misc:src-misc-helpers-module}\label{src.misc:module-src.misc.helpers}\index{src.misc.helpers (module)}\index{alt() (in module src.misc.helpers)}

\begin{fulllineitems}
\phantomsection\label{src.misc:src.misc.helpers.alt}\pysiglinewithargsret{\code{src.misc.helpers.}\bfcode{alt}}{\emph{func}}{}
Prepends the local time to the output of a function.
\begin{quote}\begin{description}
\item[{Parameters}] \leavevmode
\textbf{\texttt{func}} (\emph{\texttt{function}}) -- Function the local time should be prepended to.

\end{description}\end{quote}

\end{fulllineitems}

\index{capitalize() (in module src.misc.helpers)}

\begin{fulllineitems}
\phantomsection\label{src.misc:src.misc.helpers.capitalize}\pysiglinewithargsret{\code{src.misc.helpers.}\bfcode{capitalize}}{\emph{word}}{}
\end{fulllineitems}

\index{load\_vectors\_from\_model() (in module src.misc.helpers)}

\begin{fulllineitems}
\phantomsection\label{src.misc:src.misc.helpers.load_vectors_from_model}\pysiglinewithargsret{\code{src.misc.helpers.}\bfcode{load\_vectors\_from\_model}}{\emph{vector\_inpath}, \emph{max\_n=None}, \emph{logpath=None}, \emph{indices=False}}{}
\end{fulllineitems}



\paragraph{Module contents}
\label{src.misc:module-contents}\label{src.misc:module-src.misc}\index{src.misc (module)}

\subsubsection{src.prep package}
\label{src.prep::doc}\label{src.prep:src-prep-package}

\paragraph{Subpackages}
\label{src.prep:subpackages}

\subparagraph{src.prep.corpus package}
\label{src.prep.corpus::doc}\label{src.prep.corpus:src-prep-corpus-package}

\subparagraph{Submodules}
\label{src.prep.corpus:submodules}

\subparagraph{src.prep.corpus.convert\_to\_plain module}
\label{src.prep.corpus:src-prep-corpus-convert-to-plain-module}\label{src.prep.corpus:module-src.prep.corpus.convert_to_plain}\index{src.prep.corpus.convert\_to\_plain (module)}
Convert the \emph{DECOW14X} corpus into a plain text file. Is used as pre-processing step for the
\href{https://code.google.com/archive/p/word2vec/}{word2vec} training.
To make this this more feasible (decow is a \textbf{huge} corpus), python's \code{multiprocessing} is used, s.t. every
part of the corpus in simultaneously processed. Afterwards, a bash command like \code{cat} can be used to merge into one
single file.
\index{contains\_tag() (in module src.prep.corpus.convert\_to\_plain)}

\begin{fulllineitems}
\phantomsection\label{src.prep.corpus:src.prep.corpus.convert_to_plain.contains_tag}\pysiglinewithargsret{\code{src.prep.corpus.convert\_to\_plain.}\bfcode{contains\_tag}}{\emph{line}}{}
Checks whether the current line contains an xml tag.
\begin{quote}\begin{description}
\item[{Parameters}] \leavevmode
\textbf{\texttt{line}} (\emph{\texttt{str}}) -- Current line

\item[{Returns}] \leavevmode
Whether the current line contains an xml tag.

\item[{Return type}] \leavevmode
bool

\end{description}\end{quote}

\end{fulllineitems}

\index{convert\_decow\_to\_plain() (in module src.prep.corpus.convert\_to\_plain)}

\begin{fulllineitems}
\phantomsection\label{src.prep.corpus:src.prep.corpus.convert_to_plain.convert_decow_to_plain}\pysiglinewithargsret{\code{src.prep.corpus.convert\_to\_plain.}\bfcode{convert\_decow\_to\_plain}}{\emph{decow\_dir}, \emph{out\_dir}, \emph{log\_path}, \emph{merge\_nes}, \emph{log\_interval}}{}
Convert the whole corpus into plain text.
\begin{quote}\begin{description}
\item[{Parameters}] \leavevmode\begin{itemize}
\item {} 
\textbf{\texttt{decow\_dir}} (\emph{\texttt{str}}) -- Path to directory with decow corpus paths.

\item {} 
\textbf{\texttt{out\_dir}} (\emph{\texttt{str}}) -- Path where plain text parts should be written to.

\item {} 
\textbf{\texttt{log\_path}} (\emph{\texttt{str}}) -- Path where the log files should be written to.

\item {} 
\textbf{\texttt{merge\_nes}} (\emph{\texttt{bool}}) -- Flag to indicate whether multi-word expression should be merged with underscores.

\item {} 
\textbf{\texttt{log\_interval}} (\emph{\texttt{int}}) -- Interval to log current process state in seconds.

\end{itemize}

\end{description}\end{quote}

\end{fulllineitems}

\index{convert\_part() (in module src.prep.corpus.convert\_to\_plain)}

\begin{fulllineitems}
\phantomsection\label{src.prep.corpus:src.prep.corpus.convert_to_plain.convert_part}\pysiglinewithargsret{\code{src.prep.corpus.convert\_to\_plain.}\bfcode{convert\_part}}{\emph{argstuple}}{}
Convert a corpus part into plain text without merging multiple word entries.
\begin{quote}\begin{description}
\item[{Parameters}] \leavevmode
\textbf{\texttt{argstuple}} -- Tuple of methods arguments (\code{inpath} (\emph{str}): Path to this processes' corpus part / \code{dir\_outpath}
(\emph{str}): Path to this processes' output / \code{log\_path} (\emph{str}): Path to this processes' log / \code{interval}
(\emph{int}): Logging interval in seconds)

\end{description}\end{quote}

\end{fulllineitems}

\index{convert\_part\_merging() (in module src.prep.corpus.convert\_to\_plain)}

\begin{fulllineitems}
\phantomsection\label{src.prep.corpus:src.prep.corpus.convert_to_plain.convert_part_merging}\pysiglinewithargsret{\code{src.prep.corpus.convert\_to\_plain.}\bfcode{convert\_part\_merging}}{\emph{argstuple}}{}
Convert a corpus part into plain text and merging multiple word entries.
\begin{quote}\begin{description}
\item[{Parameters}] \leavevmode
\textbf{\texttt{argstuple}} -- Tuple of methods arguments (\code{inpath} (\emph{str}): Path to this processes' corpus part / \code{dir\_outpath}
(\emph{str}): Path to this processes' output / \code{log\_path} (\emph{str}): Path to this processes' log / \code{interval}
(\emph{int}): Logging interval in seconds)

\end{description}\end{quote}

\end{fulllineitems}

\index{extract\_named\_entity() (in module src.prep.corpus.convert\_to\_plain)}

\begin{fulllineitems}
\phantomsection\label{src.prep.corpus:src.prep.corpus.convert_to_plain.extract_named_entity}\pysiglinewithargsret{\code{src.prep.corpus.convert\_to\_plain.}\bfcode{extract\_named\_entity}}{\emph{line}}{}
Extract named entity from current line.
\begin{quote}\begin{description}
\item[{Parameters}] \leavevmode
\textbf{\texttt{line}} (\emph{\texttt{str}}) -- Current line

\item[{Returns}] \leavevmode
Extracted named entity or None if no named entity is present.

\item[{Return type}] \leavevmode
str or None

\end{description}\end{quote}

\end{fulllineitems}

\index{extract\_sentence\_id() (in module src.prep.corpus.convert\_to\_plain)}

\begin{fulllineitems}
\phantomsection\label{src.prep.corpus:src.prep.corpus.convert_to_plain.extract_sentence_id}\pysiglinewithargsret{\code{src.prep.corpus.convert\_to\_plain.}\bfcode{extract\_sentence\_id}}{\emph{tag}}{}
Extract the sentence ID of current sentence.
\begin{quote}\begin{description}
\item[{Parameters}] \leavevmode
\textbf{\texttt{tag}} (\emph{\texttt{str}}) -- Sentence tag

\item[{Returns}] \leavevmode
sentence ID

\item[{Return type}] \leavevmode
str

\end{description}\end{quote}

\end{fulllineitems}

\index{get\_file\_number() (in module src.prep.corpus.convert\_to\_plain)}

\begin{fulllineitems}
\phantomsection\label{src.prep.corpus:src.prep.corpus.convert_to_plain.get_file_number}\pysiglinewithargsret{\code{src.prep.corpus.convert\_to\_plain.}\bfcode{get\_file\_number}}{\emph{filename}}{}
Get the number of the current decow corpus part.
\begin{quote}\begin{description}
\item[{Parameters}] \leavevmode
\textbf{\texttt{filename}} (\emph{\texttt{str}}) -- Decow corpus part file name

\item[{Returns}] \leavevmode
File number

\item[{Return type}] \leavevmode
str

\end{description}\end{quote}

\end{fulllineitems}

\index{main() (in module src.prep.corpus.convert\_to\_plain)}

\begin{fulllineitems}
\phantomsection\label{src.prep.corpus:src.prep.corpus.convert_to_plain.main}\pysiglinewithargsret{\code{src.prep.corpus.convert\_to\_plain.}\bfcode{main}}{}{}
Main function. Uses command lines to start corpus processing.

\end{fulllineitems}



\subparagraph{src.prep.corpus.extract\_conll module}
\label{src.prep.corpus:src-prep-corpus-extract-conll-module}\label{src.prep.corpus:module-src.prep.corpus.extract_conll}\index{src.prep.corpus.extract\_conll (module)}
This script can be used to extract information out of a specific column of a file in the
\href{http://ilk.uvt.nl/conll/}{CoNLL}-format.
\index{extract\_conll() (in module src.prep.corpus.extract\_conll)}

\begin{fulllineitems}
\phantomsection\label{src.prep.corpus:src.prep.corpus.extract_conll.extract_conll}\pysiglinewithargsret{\code{src.prep.corpus.extract\_conll.}\bfcode{extract\_conll}}{\emph{inpath}, \emph{outpath}, \emph{column}}{}
Extract information out of CoNLL files.
\begin{quote}\begin{description}
\item[{Parameters}] \leavevmode\begin{itemize}
\item {} 
\textbf{\texttt{inpath}} (\emph{\texttt{str}}) -- Path to input file.

\item {} 
\textbf{\texttt{outpath}} (\emph{\texttt{str}}) -- Path to output file.

\item {} 
\textbf{\texttt{column}} (\emph{\texttt{int}}) -- The number (-1) of the column the information should be extracted from.

\end{itemize}

\end{description}\end{quote}

\end{fulllineitems}

\index{init\_argparse() (in module src.prep.corpus.extract\_conll)}

\begin{fulllineitems}
\phantomsection\label{src.prep.corpus:src.prep.corpus.extract_conll.init_argparse}\pysiglinewithargsret{\code{src.prep.corpus.extract\_conll.}\bfcode{init\_argparse}}{}{}
Initialize all possible arguments for the argument parser.
\begin{quote}\begin{description}
\item[{Returns}] \leavevmode
ArgumentParser object with command line arguments for this script.

\item[{Return type}] \leavevmode
\code{argparse.ArgumentParser}

\end{description}\end{quote}

\end{fulllineitems}

\index{main() (in module src.prep.corpus.extract\_conll)}

\begin{fulllineitems}
\phantomsection\label{src.prep.corpus:src.prep.corpus.extract_conll.main}\pysiglinewithargsret{\code{src.prep.corpus.extract\_conll.}\bfcode{main}}{}{}
The main function.

\end{fulllineitems}



\subparagraph{src.prep.corpus.mapper module}
\label{src.prep.corpus:module-src.prep.corpus.mapper}\label{src.prep.corpus:src-prep-corpus-mapper-module}\index{src.prep.corpus.mapper (module)}
Mapper classed used to count frequencies of words in a corpus. Corpus has to be in plain text format. This class is
used in a \href{https://en.wikipedia.org/wiki/MapReduce}{Map-Reduce}-pattern, so you also need the \code{reducer.py} class.

Then, you can open your terminal and pipe them together:

\begin{Verbatim}[commandchars=\\\{\}]
\PYG{g+gp}{\PYGZgt{}} cat corpus.txt \PYG{p}{\textbar{}} ./mapper.py \PYG{p}{\textbar{}} sort \PYG{p}{\textbar{}} ./reducer.py
\end{Verbatim}

Also, you probably have to remove the \code{if \_\_name\_\_ == "\_\_main\_\_":} line and unindent the remaining code,
this is only
due to sphinx being picky and not documenting plain python scripts at all.


\subparagraph{src.prep.corpus.reducer module}
\label{src.prep.corpus:module-src.prep.corpus.reducer}\label{src.prep.corpus:src-prep-corpus-reducer-module}\index{src.prep.corpus.reducer (module)}
Reducer classed used to count frequencies of words in a corpus. Corpus has to be in plain text format. This class is
used in a \href{https://en.wikipedia.org/wiki/MapReduce}{Map-Reduce}-pattern, so you also need the \code{mapper.py} class.

Then, you can open your terminal and pipe them together:

\begin{Verbatim}[commandchars=\\\{\}]
\PYG{g+gp}{\PYGZgt{}} cat corpus.txt \PYG{p}{\textbar{}} ./mapper.py \PYG{p}{\textbar{}} sort \PYG{p}{\textbar{}} ./reducer.py
\end{Verbatim}

Also, you probably have to remove the \code{if \_\_name\_\_ == "\_\_main\_\_":} line and unindent the remaining code,
this is only
due to sphinx being picky and not documenting plain python scripts at all.


\subparagraph{Module contents}
\label{src.prep.corpus:module-contents}\label{src.prep.corpus:module-src.prep.corpus}\index{src.prep.corpus (module)}

\subparagraph{src.prep.nes package}
\label{src.prep.nes:src-prep-nes-package}\label{src.prep.nes::doc}

\subparagraph{Submodules}
\label{src.prep.nes:submodules}

\subparagraph{src.prep.nes.extractNE module}
\label{src.prep.nes:src-prep-nes-extractne-module}\label{src.prep.nes:module-src.prep.nes.extractNE}\index{src.prep.nes.extractNE (module)}\index{contains\_tag() (in module src.prep.nes.extractNE)}

\begin{fulllineitems}
\phantomsection\label{src.prep.nes:src.prep.nes.extractNE.contains_tag}\pysiglinewithargsret{\code{src.prep.nes.extractNE.}\bfcode{contains\_tag}}{\emph{line}}{}
\end{fulllineitems}

\index{extract\_named\_entity() (in module src.prep.nes.extractNE)}

\begin{fulllineitems}
\phantomsection\label{src.prep.nes:src.prep.nes.extractNE.extract_named_entity}\pysiglinewithargsret{\code{src.prep.nes.extractNE.}\bfcode{extract\_named\_entity}}{\emph{line}}{}
\end{fulllineitems}

\index{extract\_sentence\_id() (in module src.prep.nes.extractNE)}

\begin{fulllineitems}
\phantomsection\label{src.prep.nes:src.prep.nes.extractNE.extract_sentence_id}\pysiglinewithargsret{\code{src.prep.nes.extractNE.}\bfcode{extract\_sentence\_id}}{\emph{tag}}{}
\end{fulllineitems}

\index{main() (in module src.prep.nes.extractNE)}

\begin{fulllineitems}
\phantomsection\label{src.prep.nes:src.prep.nes.extractNE.main}\pysiglinewithargsret{\code{src.prep.nes.extractNE.}\bfcode{main}}{}{}
\end{fulllineitems}

\index{print\_dict\_in\_file() (in module src.prep.nes.extractNE)}

\begin{fulllineitems}
\phantomsection\label{src.prep.nes:src.prep.nes.extractNE.print_dict_in_file}\pysiglinewithargsret{\code{src.prep.nes.extractNE.}\bfcode{print\_dict\_in\_file}}{\emph{dictionary}, \emph{out\_path}}{}
\end{fulllineitems}

\index{print\_ids\_in\_file() (in module src.prep.nes.extractNE)}

\begin{fulllineitems}
\phantomsection\label{src.prep.nes:src.prep.nes.extractNE.print_ids_in_file}\pysiglinewithargsret{\code{src.prep.nes.extractNE.}\bfcode{print\_ids\_in\_file}}{\emph{dictionary}, \emph{out\_path}}{}
\end{fulllineitems}

\index{print\_list\_in\_file() (in module src.prep.nes.extractNE)}

\begin{fulllineitems}
\phantomsection\label{src.prep.nes:src.prep.nes.extractNE.print_list_in_file}\pysiglinewithargsret{\code{src.prep.nes.extractNE.}\bfcode{print\_list\_in\_file}}{\emph{ne\_list}, \emph{out\_path}}{}
\end{fulllineitems}

\index{process() (in module src.prep.nes.extractNE)}

\begin{fulllineitems}
\phantomsection\label{src.prep.nes:src.prep.nes.extractNE.process}\pysiglinewithargsret{\code{src.prep.nes.extractNE.}\bfcode{process}}{\emph{inpath}, \emph{outpath}, \emph{logpath}}{}
\end{fulllineitems}



\subparagraph{src.prep.nes.merge module}
\label{src.prep.nes:src-prep-nes-merge-module}\label{src.prep.nes:module-src.prep.nes.merge}\index{src.prep.nes.merge (module)}\index{dump\_ids\_dict() (in module src.prep.nes.merge)}

\begin{fulllineitems}
\phantomsection\label{src.prep.nes:src.prep.nes.merge.dump_ids_dict}\pysiglinewithargsret{\code{src.prep.nes.merge.}\bfcode{dump\_ids\_dict}}{\emph{idsdict}, \emph{outpath}}{}
\end{fulllineitems}

\index{freqWorker() (in module src.prep.nes.merge)}

\begin{fulllineitems}
\phantomsection\label{src.prep.nes:src.prep.nes.merge.freqWorker}\pysiglinewithargsret{\code{src.prep.nes.merge.}\bfcode{freqWorker}}{\emph{inpath}}{}
\end{fulllineitems}

\index{idWorker() (in module src.prep.nes.merge)}

\begin{fulllineitems}
\phantomsection\label{src.prep.nes:src.prep.nes.merge.idWorker}\pysiglinewithargsret{\code{src.prep.nes.merge.}\bfcode{idWorker}}{\emph{inpath}}{}
\end{fulllineitems}

\index{load\_ids\_dict() (in module src.prep.nes.merge)}

\begin{fulllineitems}
\phantomsection\label{src.prep.nes:src.prep.nes.merge.load_ids_dict}\pysiglinewithargsret{\code{src.prep.nes.merge.}\bfcode{load\_ids\_dict}}{\emph{inpath}}{}
\end{fulllineitems}

\index{main() (in module src.prep.nes.merge)}

\begin{fulllineitems}
\phantomsection\label{src.prep.nes:src.prep.nes.merge.main}\pysiglinewithargsret{\code{src.prep.nes.merge.}\bfcode{main}}{}{}
\end{fulllineitems}

\index{mergeDicts() (in module src.prep.nes.merge)}

\begin{fulllineitems}
\phantomsection\label{src.prep.nes:src.prep.nes.merge.mergeDicts}\pysiglinewithargsret{\code{src.prep.nes.merge.}\bfcode{mergeDicts}}{\emph{dicttuple}}{}
\end{fulllineitems}

\index{merge\_frequency\_files() (in module src.prep.nes.merge)}

\begin{fulllineitems}
\phantomsection\label{src.prep.nes:src.prep.nes.merge.merge_frequency_files}\pysiglinewithargsret{\code{src.prep.nes.merge.}\bfcode{merge\_frequency\_files}}{\emph{infiles\_path}, \emph{outpath}, \emph{logpath}}{}
\end{fulllineitems}

\index{merge\_id\_dicts() (in module src.prep.nes.merge)}

\begin{fulllineitems}
\phantomsection\label{src.prep.nes:src.prep.nes.merge.merge_id_dicts}\pysiglinewithargsret{\code{src.prep.nes.merge.}\bfcode{merge\_id\_dicts}}{\emph{dicttuple}}{}
\end{fulllineitems}

\index{merge\_id\_files() (in module src.prep.nes.merge)}

\begin{fulllineitems}
\phantomsection\label{src.prep.nes:src.prep.nes.merge.merge_id_files}\pysiglinewithargsret{\code{src.prep.nes.merge.}\bfcode{merge\_id\_files}}{\emph{infiles\_path}, \emph{outpath}, \emph{logpath}, \emph{yaml=False}}{}
\end{fulllineitems}

\index{print\_key\_lengths() (in module src.prep.nes.merge)}

\begin{fulllineitems}
\phantomsection\label{src.prep.nes:src.prep.nes.merge.print_key_lengths}\pysiglinewithargsret{\code{src.prep.nes.merge.}\bfcode{print\_key\_lengths}}{\emph{dictionary}}{}
\end{fulllineitems}

\index{rl() (in module src.prep.nes.merge)}

\begin{fulllineitems}
\phantomsection\label{src.prep.nes:src.prep.nes.merge.rl}\pysiglinewithargsret{\code{src.prep.nes.merge.}\bfcode{rl}}{\emph{infile}}{}
\end{fulllineitems}



\subparagraph{src.prep.nes.mwe module}
\label{src.prep.nes:module-src.prep.nes.mwe}\label{src.prep.nes:src-prep-nes-mwe-module}\index{src.prep.nes.mwe (module)}\index{create\_mwe\_pickle() (in module src.prep.nes.mwe)}

\begin{fulllineitems}
\phantomsection\label{src.prep.nes:src.prep.nes.mwe.create_mwe_pickle}\pysiglinewithargsret{\code{src.prep.nes.mwe.}\bfcode{create\_mwe\_pickle}}{\emph{inpath}, \emph{outpath}, \emph{logpath='./mwes.log'}}{}
\end{fulllineitems}

\index{create\_mwe\_pickle2() (in module src.prep.nes.mwe)}

\begin{fulllineitems}
\phantomsection\label{src.prep.nes:src.prep.nes.mwe.create_mwe_pickle2}\pysiglinewithargsret{\code{src.prep.nes.mwe.}\bfcode{create\_mwe\_pickle2}}{\emph{inpath}, \emph{outpath}, \emph{logpath='./mwes.log'}}{}
\end{fulllineitems}

\index{dump\_dict\_pickle() (in module src.prep.nes.mwe)}

\begin{fulllineitems}
\phantomsection\label{src.prep.nes:src.prep.nes.mwe.dump_dict_pickle}\pysiglinewithargsret{\code{src.prep.nes.mwe.}\bfcode{dump\_dict\_pickle}}{\emph{d}, \emph{outpath}}{}
\end{fulllineitems}

\index{dump\_dict\_pickle2() (in module src.prep.nes.mwe)}

\begin{fulllineitems}
\phantomsection\label{src.prep.nes:src.prep.nes.mwe.dump_dict_pickle2}\pysiglinewithargsret{\code{src.prep.nes.mwe.}\bfcode{dump\_dict\_pickle2}}{\emph{d}, \emph{outpath}}{}
\end{fulllineitems}

\index{load\_dict\_pickle() (in module src.prep.nes.mwe)}

\begin{fulllineitems}
\phantomsection\label{src.prep.nes:src.prep.nes.mwe.load_dict_pickle}\pysiglinewithargsret{\code{src.prep.nes.mwe.}\bfcode{load\_dict\_pickle}}{\emph{inpath}}{}
\end{fulllineitems}

\index{load\_dict\_pickle2() (in module src.prep.nes.mwe)}

\begin{fulllineitems}
\phantomsection\label{src.prep.nes:src.prep.nes.mwe.load_dict_pickle2}\pysiglinewithargsret{\code{src.prep.nes.mwe.}\bfcode{load\_dict\_pickle2}}{\emph{inpath}}{}
\end{fulllineitems}

\index{main() (in module src.prep.nes.mwe)}

\begin{fulllineitems}
\phantomsection\label{src.prep.nes:src.prep.nes.mwe.main}\pysiglinewithargsret{\code{src.prep.nes.mwe.}\bfcode{main}}{}{}
\end{fulllineitems}

\index{replace\_mwes() (in module src.prep.nes.mwe)}

\begin{fulllineitems}
\phantomsection\label{src.prep.nes:src.prep.nes.mwe.replace_mwes}\pysiglinewithargsret{\code{src.prep.nes.mwe.}\bfcode{replace\_mwes}}{\emph{mwe\_path}, \emph{corpus\_path}, \emph{out\_path}}{}
\end{fulllineitems}



\subparagraph{src.prep.nes.statistics module}
\label{src.prep.nes:src-prep-nes-statistics-module}\label{src.prep.nes:module-src.prep.nes.statistics}\index{src.prep.nes.statistics (module)}\index{calculate\_occurrences() (in module src.prep.nes.statistics)}

\begin{fulllineitems}
\phantomsection\label{src.prep.nes:src.prep.nes.statistics.calculate_occurrences}\pysiglinewithargsret{\code{src.prep.nes.statistics.}\bfcode{calculate\_occurrences}}{\emph{freqpath}, \emph{relations\_path}}{}
\end{fulllineitems}

\index{main() (in module src.prep.nes.statistics)}

\begin{fulllineitems}
\phantomsection\label{src.prep.nes:src.prep.nes.statistics.main}\pysiglinewithargsret{\code{src.prep.nes.statistics.}\bfcode{main}}{}{}
\end{fulllineitems}



\subparagraph{Module contents}
\label{src.prep.nes:module-src.prep.nes}\label{src.prep.nes:module-contents}\index{src.prep.nes (module)}

\subparagraph{src.prep.relations package}
\label{src.prep.relations:src-prep-relations-package}\label{src.prep.relations::doc}

\subparagraph{Submodules}
\label{src.prep.relations:submodules}

\subparagraph{src.prep.relations.relations module}
\label{src.prep.relations:src-prep-relations-relations-module}\label{src.prep.relations:module-src.prep.relations.relations}\index{src.prep.relations.relations (module)}\index{MissingTranslationException}

\begin{fulllineitems}
\phantomsection\label{src.prep.relations:src.prep.relations.relations.MissingTranslationException}\pysigline{\strong{exception }\code{src.prep.relations.relations.}\bfcode{MissingTranslationException}}
Bases: \code{exceptions.Exception}
\index{get\_id() (src.prep.relations.relations.MissingTranslationException method)}

\begin{fulllineitems}
\phantomsection\label{src.prep.relations:src.prep.relations.relations.MissingTranslationException.get_id}\pysiglinewithargsret{\bfcode{get\_id}}{}{}
\end{fulllineitems}


\end{fulllineitems}

\index{fetch\_name() (in module src.prep.relations.relations)}

\begin{fulllineitems}
\phantomsection\label{src.prep.relations:src.prep.relations.relations.fetch_name}\pysiglinewithargsret{\code{src.prep.relations.relations.}\bfcode{fetch\_name}}{\emph{id}, \emph{lang='en'}}{}
\end{fulllineitems}

\index{fetch\_relation\_triples\_of\_file() (in module src.prep.relations.relations)}

\begin{fulllineitems}
\phantomsection\label{src.prep.relations:src.prep.relations.relations.fetch_relation_triples_of_file}\pysiglinewithargsret{\code{src.prep.relations.relations.}\bfcode{fetch\_relation\_triples\_of\_file}}{\emph{inpath}, \emph{outpath}, \emph{logpath}, \emph{lang='en'}}{}
\end{fulllineitems}

\index{format\_fbid() (in module src.prep.relations.relations)}

\begin{fulllineitems}
\phantomsection\label{src.prep.relations:src.prep.relations.relations.format_fbid}\pysiglinewithargsret{\code{src.prep.relations.relations.}\bfcode{format\_fbid}}{\emph{id}}{}
\end{fulllineitems}

\index{freebase\_request() (in module src.prep.relations.relations)}

\begin{fulllineitems}
\phantomsection\label{src.prep.relations:src.prep.relations.relations.freebase_request}\pysiglinewithargsret{\code{src.prep.relations.relations.}\bfcode{freebase\_request}}{\emph{query}, \emph{api\_key}, \emph{service\_url}}{}
\end{fulllineitems}

\index{main() (in module src.prep.relations.relations)}

\begin{fulllineitems}
\phantomsection\label{src.prep.relations:src.prep.relations.relations.main}\pysiglinewithargsret{\code{src.prep.relations.relations.}\bfcode{main}}{}{}
\end{fulllineitems}

\index{read\_credentials() (in module src.prep.relations.relations)}

\begin{fulllineitems}
\phantomsection\label{src.prep.relations:src.prep.relations.relations.read_credentials}\pysiglinewithargsret{\code{src.prep.relations.relations.}\bfcode{read\_credentials}}{}{}
\end{fulllineitems}

\index{rl() (in module src.prep.relations.relations)}

\begin{fulllineitems}
\phantomsection\label{src.prep.relations:src.prep.relations.relations.rl}\pysiglinewithargsret{\code{src.prep.relations.relations.}\bfcode{rl}}{\emph{infile}}{}
\end{fulllineitems}

\index{translate\_name() (in module src.prep.relations.relations)}

\begin{fulllineitems}
\phantomsection\label{src.prep.relations:src.prep.relations.relations.translate_name}\pysiglinewithargsret{\code{src.prep.relations.relations.}\bfcode{translate\_name}}{\emph{name}, \emph{lang='en'}}{}
\end{fulllineitems}

\index{translate\_word2vec\_question\_phrases() (in module src.prep.relations.relations)}

\begin{fulllineitems}
\phantomsection\label{src.prep.relations:src.prep.relations.relations.translate_word2vec_question_phrases}\pysiglinewithargsret{\code{src.prep.relations.relations.}\bfcode{translate\_word2vec\_question\_phrases}}{\emph{inpath}, \emph{outpath}, \emph{lang='en'}}{}
\end{fulllineitems}



\subparagraph{Module contents}
\label{src.prep.relations:module-src.prep.relations}\label{src.prep.relations:module-contents}\index{src.prep.relations (module)}

\paragraph{Module contents}
\label{src.prep:module-contents}\label{src.prep:module-src.prep}\index{src.prep (module)}

\subsubsection{src.trans\_e package}
\label{src.trans_e:src-trans-e-package}\label{src.trans_e::doc}

\paragraph{Submodules}
\label{src.trans_e:submodules}

\paragraph{src.trans\_e.add\_inverse\_relations module}
\label{src.trans_e:module-src.trans_e.add_inverse_relations}\label{src.trans_e:src-trans-e-add-inverse-relations-module}\index{src.trans\_e.add\_inverse\_relations (module)}\index{add\_inverse\_relations() (in module src.trans\_e.add\_inverse\_relations)}

\begin{fulllineitems}
\phantomsection\label{src.trans_e:src.trans_e.add_inverse_relations.add_inverse_relations}\pysiglinewithargsret{\code{src.trans\_e.add\_inverse\_relations.}\bfcode{add\_inverse\_relations}}{\emph{relations\_inpath}, \emph{relations\_outpath}, \emph{inverse\_relations}, \emph{known\_relations}}{}
\end{fulllineitems}

\index{init\_argparse() (in module src.trans\_e.add\_inverse\_relations)}

\begin{fulllineitems}
\phantomsection\label{src.trans_e:src.trans_e.add_inverse_relations.init_argparse}\pysiglinewithargsret{\code{src.trans\_e.add\_inverse\_relations.}\bfcode{init\_argparse}}{}{}
\end{fulllineitems}

\index{main() (in module src.trans\_e.add\_inverse\_relations)}

\begin{fulllineitems}
\phantomsection\label{src.trans_e:src.trans_e.add_inverse_relations.main}\pysiglinewithargsret{\code{src.trans\_e.add\_inverse\_relations.}\bfcode{main}}{}{}
\end{fulllineitems}

\index{read\_file\_with\_inverse\_relations() (in module src.trans\_e.add\_inverse\_relations)}

\begin{fulllineitems}
\phantomsection\label{src.trans_e:src.trans_e.add_inverse_relations.read_file_with_inverse_relations}\pysiglinewithargsret{\code{src.trans\_e.add\_inverse\_relations.}\bfcode{read\_file\_with\_inverse\_relations}}{\emph{inverse\_inpath}}{}
\end{fulllineitems}



\paragraph{src.trans\_e.clean\_relations module}
\label{src.trans_e:src-trans-e-clean-relations-module}\label{src.trans_e:module-src.trans_e.clean_relations}\index{src.trans\_e.clean\_relations (module)}

\paragraph{src.trans\_e.contains\_entities module}
\label{src.trans_e:src-trans-e-contains-entities-module}\label{src.trans_e:module-src.trans_e.contains_entities}\index{src.trans\_e.contains\_entities (module)}\index{contains\_entities() (in module src.trans\_e.contains\_entities)}

\begin{fulllineitems}
\phantomsection\label{src.trans_e:src.trans_e.contains_entities.contains_entities}\pysiglinewithargsret{\code{src.trans\_e.contains\_entities.}\bfcode{contains\_entities}}{\emph{entities1}, \emph{entities2}}{}
\end{fulllineitems}

\index{create\_new\_dataset() (in module src.trans\_e.contains\_entities)}

\begin{fulllineitems}
\phantomsection\label{src.trans_e:src.trans_e.contains_entities.create_new_dataset}\pysiglinewithargsret{\code{src.trans\_e.contains\_entities.}\bfcode{create\_new\_dataset}}{\emph{entities1}, \emph{dataset}, \emph{outpath}}{}
\end{fulllineitems}

\index{extract\_entities\_from\_relation\_dataset() (in module src.trans\_e.contains\_entities)}

\begin{fulllineitems}
\phantomsection\label{src.trans_e:src.trans_e.contains_entities.extract_entities_from_relation_dataset}\pysiglinewithargsret{\code{src.trans\_e.contains\_entities.}\bfcode{extract\_entities\_from\_relation\_dataset}}{\emph{dataset\_inpath}}{}
\end{fulllineitems}

\index{extract\_entities\_from\_tql\_file() (in module src.trans\_e.contains\_entities)}

\begin{fulllineitems}
\phantomsection\label{src.trans_e:src.trans_e.contains_entities.extract_entities_from_tql_file}\pysiglinewithargsret{\code{src.trans\_e.contains\_entities.}\bfcode{extract\_entities\_from\_tql\_file}}{\emph{tql\_path}}{}
\end{fulllineitems}

\index{format\_fbid() (in module src.trans\_e.contains\_entities)}

\begin{fulllineitems}
\phantomsection\label{src.trans_e:src.trans_e.contains_entities.format_fbid}\pysiglinewithargsret{\code{src.trans\_e.contains\_entities.}\bfcode{format\_fbid}}{\emph{id}}{}
\end{fulllineitems}

\index{init\_argparse() (in module src.trans\_e.contains\_entities)}

\begin{fulllineitems}
\phantomsection\label{src.trans_e:src.trans_e.contains_entities.init_argparse}\pysiglinewithargsret{\code{src.trans\_e.contains\_entities.}\bfcode{init\_argparse}}{}{}
\end{fulllineitems}

\index{main() (in module src.trans\_e.contains\_entities)}

\begin{fulllineitems}
\phantomsection\label{src.trans_e:src.trans_e.contains_entities.main}\pysiglinewithargsret{\code{src.trans\_e.contains\_entities.}\bfcode{main}}{}{}
\end{fulllineitems}



\paragraph{src.trans\_e.convert\_relations module}
\label{src.trans_e:module-src.trans_e.convert_relations}\label{src.trans_e:src-trans-e-convert-relations-module}\index{src.trans\_e.convert\_relations (module)}

\paragraph{src.trans\_e.differentiate\_datasets module}
\label{src.trans_e:module-src.trans_e.differentiate_datasets}\label{src.trans_e:src-trans-e-differentiate-datasets-module}\index{src.trans\_e.differentiate\_datasets (module)}\index{compare\_entities() (in module src.trans\_e.differentiate\_datasets)}

\begin{fulllineitems}
\phantomsection\label{src.trans_e:src.trans_e.differentiate_datasets.compare_entities}\pysiglinewithargsret{\code{src.trans\_e.differentiate\_datasets.}\bfcode{compare\_entities}}{\emph{set1}, \emph{set2}}{}
\end{fulllineitems}

\index{init\_argparse() (in module src.trans\_e.differentiate\_datasets)}

\begin{fulllineitems}
\phantomsection\label{src.trans_e:src.trans_e.differentiate_datasets.init_argparse}\pysiglinewithargsret{\code{src.trans\_e.differentiate\_datasets.}\bfcode{init\_argparse}}{}{}
\end{fulllineitems}

\index{main() (in module src.trans\_e.differentiate\_datasets)}

\begin{fulllineitems}
\phantomsection\label{src.trans_e:src.trans_e.differentiate_datasets.main}\pysiglinewithargsret{\code{src.trans\_e.differentiate\_datasets.}\bfcode{main}}{}{}
\end{fulllineitems}

\index{read\_dataset() (in module src.trans\_e.differentiate\_datasets)}

\begin{fulllineitems}
\phantomsection\label{src.trans_e:src.trans_e.differentiate_datasets.read_dataset}\pysiglinewithargsret{\code{src.trans\_e.differentiate\_datasets.}\bfcode{read\_dataset}}{\emph{inpath}}{}
\end{fulllineitems}



\paragraph{src.trans\_e.partition\_data module}
\label{src.trans_e:src-trans-e-partition-data-module}\label{src.trans_e:module-src.trans_e.partition_data}\index{src.trans\_e.partition\_data (module)}\index{check\_data\_integrity() (in module src.trans\_e.partition\_data)}

\begin{fulllineitems}
\phantomsection\label{src.trans_e:src.trans_e.partition_data.check_data_integrity}\pysiglinewithargsret{\code{src.trans\_e.partition\_data.}\bfcode{check\_data\_integrity}}{\emph{data\_inpath}, \emph{remove\_clones}, \emph{outpath}}{}
Check whether all triplets in the data are unique.

\end{fulllineitems}

\index{check\_set\_integrity() (in module src.trans\_e.partition\_data)}

\begin{fulllineitems}
\phantomsection\label{src.trans_e:src.trans_e.partition_data.check_set_integrity}\pysiglinewithargsret{\code{src.trans\_e.partition\_data.}\bfcode{check\_set\_integrity}}{\emph{indir}}{}
\end{fulllineitems}

\index{get\_stats() (in module src.trans\_e.partition\_data)}

\begin{fulllineitems}
\phantomsection\label{src.trans_e:src.trans_e.partition_data.get_stats}\pysiglinewithargsret{\code{src.trans\_e.partition\_data.}\bfcode{get\_stats}}{\emph{data}}{}
\end{fulllineitems}

\index{init\_argparse() (in module src.trans\_e.partition\_data)}

\begin{fulllineitems}
\phantomsection\label{src.trans_e:src.trans_e.partition_data.init_argparse}\pysiglinewithargsret{\code{src.trans\_e.partition\_data.}\bfcode{init\_argparse}}{}{}
\end{fulllineitems}

\index{main() (in module src.trans\_e.partition\_data)}

\begin{fulllineitems}
\phantomsection\label{src.trans_e:src.trans_e.partition_data.main}\pysiglinewithargsret{\code{src.trans\_e.partition\_data.}\bfcode{main}}{}{}
\end{fulllineitems}

\index{partition\_data() (in module src.trans\_e.partition\_data)}

\begin{fulllineitems}
\phantomsection\label{src.trans_e:src.trans_e.partition_data.partition_data}\pysiglinewithargsret{\code{src.trans\_e.partition\_data.}\bfcode{partition\_data}}{\emph{data}, \emph{prts}, \emph{outdir}, \emph{whole=True}}{}
\end{fulllineitems}

\index{partition\_relation\_wise() (in module src.trans\_e.partition\_data)}

\begin{fulllineitems}
\phantomsection\label{src.trans_e:src.trans_e.partition_data.partition_relation_wise}\pysiglinewithargsret{\code{src.trans\_e.partition\_data.}\bfcode{partition\_relation\_wise}}{\emph{data}, \emph{prts}}{}
\end{fulllineitems}

\index{partition\_whole() (in module src.trans\_e.partition\_data)}

\begin{fulllineitems}
\phantomsection\label{src.trans_e:src.trans_e.partition_data.partition_whole}\pysiglinewithargsret{\code{src.trans\_e.partition\_data.}\bfcode{partition\_whole}}{\emph{data}, \emph{prts}}{}
\end{fulllineitems}

\index{partitions\_list() (in module src.trans\_e.partition\_data)}

\begin{fulllineitems}
\phantomsection\label{src.trans_e:src.trans_e.partition_data.partitions_list}\pysiglinewithargsret{\code{src.trans\_e.partition\_data.}\bfcode{partitions\_list}}{\emph{l}, \emph{prts}}{}
\end{fulllineitems}

\index{read\_only\_relations\_into\_set() (in module src.trans\_e.partition\_data)}

\begin{fulllineitems}
\phantomsection\label{src.trans_e:src.trans_e.partition_data.read_only_relations_into_set}\pysiglinewithargsret{\code{src.trans\_e.partition\_data.}\bfcode{read\_only\_relations\_into\_set}}{\emph{inpath}}{}
\end{fulllineitems}

\index{read\_relations() (in module src.trans\_e.partition\_data)}

\begin{fulllineitems}
\phantomsection\label{src.trans_e:src.trans_e.partition_data.read_relations}\pysiglinewithargsret{\code{src.trans\_e.partition\_data.}\bfcode{read\_relations}}{\emph{inpath}}{}
\end{fulllineitems}

\index{write\_data\_in\_file() (in module src.trans\_e.partition\_data)}

\begin{fulllineitems}
\phantomsection\label{src.trans_e:src.trans_e.partition_data.write_data_in_file}\pysiglinewithargsret{\code{src.trans\_e.partition\_data.}\bfcode{write\_data\_in\_file}}{\emph{data}, \emph{outfile}}{}
\end{fulllineitems}



\paragraph{src.trans\_e.trans\_we module}
\label{src.trans_e:src-trans-e-trans-we-module}\label{src.trans_e:module-src.trans_e.trans_we}\index{src.trans\_e.trans\_we (module)}\index{convert\_data() (in module src.trans\_e.trans\_we)}

\begin{fulllineitems}
\phantomsection\label{src.trans_e:src.trans_e.trans_we.convert_data}\pysiglinewithargsret{\code{src.trans\_e.trans\_we.}\bfcode{convert\_data}}{\emph{sets\_path}, \emph{tql\_inpath}, \emph{vector\_inpath}}{}
\end{fulllineitems}

\index{create\_corrupt\_triples() (in module src.trans\_e.trans\_we)}

\begin{fulllineitems}
\phantomsection\label{src.trans_e:src.trans_e.trans_we.create_corrupt_triples}\pysiglinewithargsret{\code{src.trans\_e.trans\_we.}\bfcode{create\_corrupt\_triples}}{\emph{grouped\_pairs}, \emph{entities}}{}
\end{fulllineitems}

\index{dump\_relation\_vectors() (in module src.trans\_e.trans\_we)}

\begin{fulllineitems}
\phantomsection\label{src.trans_e:src.trans_e.trans_we.dump_relation_vectors}\pysiglinewithargsret{\code{src.trans\_e.trans\_we.}\bfcode{dump\_relation\_vectors}}{\emph{relation\_vectors}, \emph{outpath}}{}
\end{fulllineitems}

\index{evaluate() (in module src.trans\_e.trans\_we)}

\begin{fulllineitems}
\phantomsection\label{src.trans_e:src.trans_e.trans_we.evaluate}\pysiglinewithargsret{\code{src.trans\_e.trans\_we.}\bfcode{evaluate}}{\emph{model}, \emph{grouped\_test}, \emph{relation\_vectors}, \emph{entities}}{}
\end{fulllineitems}

\index{extract\_data\_from\_uri() (in module src.trans\_e.trans\_we)}

\begin{fulllineitems}
\phantomsection\label{src.trans_e:src.trans_e.trans_we.extract_data_from_uri}\pysiglinewithargsret{\code{src.trans\_e.trans\_we.}\bfcode{extract\_data\_from\_uri}}{\emph{uri}}{}
\end{fulllineitems}

\index{get\_rank() (in module src.trans\_e.trans\_we)}

\begin{fulllineitems}
\phantomsection\label{src.trans_e:src.trans_e.trans_we.get_rank}\pysiglinewithargsret{\code{src.trans\_e.trans\_we.}\bfcode{get\_rank}}{\emph{target}, \emph{ranks}}{}
\end{fulllineitems}

\index{init\_argparser() (in module src.trans\_e.trans\_we)}

\begin{fulllineitems}
\phantomsection\label{src.trans_e:src.trans_e.trans_we.init_argparser}\pysiglinewithargsret{\code{src.trans\_e.trans\_we.}\bfcode{init\_argparser}}{}{}
Initialize all arguments for an ArgumentParser object and return it.

@returns \{ArgumentParser\} argument parser object

\end{fulllineitems}

\index{load\_relation\_vectors() (in module src.trans\_e.trans\_we)}

\begin{fulllineitems}
\phantomsection\label{src.trans_e:src.trans_e.trans_we.load_relation_vectors}\pysiglinewithargsret{\code{src.trans\_e.trans\_we.}\bfcode{load\_relation\_vectors}}{\emph{inpath}}{}
\end{fulllineitems}

\index{load\_vectors() (in module src.trans\_e.trans\_we)}

\begin{fulllineitems}
\phantomsection\label{src.trans_e:src.trans_e.trans_we.load_vectors}\pysiglinewithargsret{\code{src.trans\_e.trans\_we.}\bfcode{load\_vectors}}{\emph{vector\_inpath}}{}
@param vector\_inpath: Path to word2vec model file

\end{fulllineitems}

\index{main() (in module src.trans\_e.trans\_we)}

\begin{fulllineitems}
\phantomsection\label{src.trans_e:src.trans_e.trans_we.main}\pysiglinewithargsret{\code{src.trans\_e.trans\_we.}\bfcode{main}}{}{}
\end{fulllineitems}

\index{prepare\_training() (in module src.trans\_e.trans\_we)}

\begin{fulllineitems}
\phantomsection\label{src.trans_e:src.trans_e.trans_we.prepare_training}\pysiglinewithargsret{\code{src.trans\_e.trans\_we.}\bfcode{prepare\_training}}{\emph{sets\_path}, \emph{vector\_inpath}}{}
\end{fulllineitems}

\index{rank\_entities() (in module src.trans\_e.trans\_we)}

\begin{fulllineitems}
\phantomsection\label{src.trans_e:src.trans_e.trans_we.rank_entities}\pysiglinewithargsret{\code{src.trans\_e.trans\_we.}\bfcode{rank\_entities}}{\emph{reference}, \emph{solution}, \emph{model}, \emph{entities}}{}
\end{fulllineitems}

\index{read\_freebase\_data() (in module src.trans\_e.trans\_we)}

\begin{fulllineitems}
\phantomsection\label{src.trans_e:src.trans_e.trans_we.read_freebase_data}\pysiglinewithargsret{\code{src.trans\_e.trans\_we.}\bfcode{read\_freebase\_data}}{\emph{sets\_path}}{}
\end{fulllineitems}

\index{read\_freebase\_file() (in module src.trans\_e.trans\_we)}

\begin{fulllineitems}
\phantomsection\label{src.trans_e:src.trans_e.trans_we.read_freebase_file}\pysiglinewithargsret{\code{src.trans\_e.trans\_we.}\bfcode{read\_freebase\_file}}{\emph{fb\_inpath}}{}
\end{fulllineitems}

\index{read\_tql\_file() (in module src.trans\_e.trans\_we)}

\begin{fulllineitems}
\phantomsection\label{src.trans_e:src.trans_e.trans_we.read_tql_file}\pysiglinewithargsret{\code{src.trans\_e.trans\_we.}\bfcode{read\_tql\_file}}{\emph{tql\_inpath}}{}
\end{fulllineitems}

\index{test\_coverage() (in module src.trans\_e.trans\_we)}

\begin{fulllineitems}
\phantomsection\label{src.trans_e:src.trans_e.trans_we.test_coverage}\pysiglinewithargsret{\code{src.trans\_e.trans\_we.}\bfcode{test\_coverage}}{\emph{triples}, \emph{model}}{}
Test the coverage of a dataset consisting of freebase triples on word2vec word embeddings.
For every triple (h, l, t), the entities h and t are taken and used for look up in the word2vec
model.

@param triples: list of 3-tuples (freebase triples)
@param model: gensim word2vec model

\end{fulllineitems}

\index{train() (in module src.trans\_e.trans\_we)}

\begin{fulllineitems}
\phantomsection\label{src.trans_e:src.trans_e.trans_we.train}\pysiglinewithargsret{\code{src.trans\_e.trans\_we.}\bfcode{train}}{\emph{model}, \emph{grouped\_train}, \emph{grouped\_corrupted}, \emph{lossf}, \emph{relation\_types}, \emph{epochs=1000}, \emph{learning\_rate=0.01}, \emph{margin=1}}{}
\end{fulllineitems}

\index{transform\_triples() (in module src.trans\_e.trans\_we)}

\begin{fulllineitems}
\phantomsection\label{src.trans_e:src.trans_e.trans_we.transform_triples}\pysiglinewithargsret{\code{src.trans\_e.trans\_we.}\bfcode{transform\_triples}}{\emph{triples}, \emph{relation\_types}, \emph{entities}}{}
\end{fulllineitems}

\index{write\_data() (in module src.trans\_e.trans\_we)}

\begin{fulllineitems}
\phantomsection\label{src.trans_e:src.trans_e.trans_we.write_data}\pysiglinewithargsret{\code{src.trans\_e.trans\_we.}\bfcode{write\_data}}{\emph{triples}, \emph{found\_entities}, \emph{outpath}}{}
\end{fulllineitems}



\paragraph{Module contents}
\label{src.trans_e:module-src.trans_e}\label{src.trans_e:module-contents}\index{src.trans\_e (module)}

\subsection{Module contents}
\label{src:module-src}\label{src:module-contents}\index{src (module)}

\chapter{Indices and tables}
\label{index:indices-and-tables}\begin{itemize}
\item {} 
\DUrole{xref,std,std-ref}{genindex}

\item {} 
\DUrole{xref,std,std-ref}{modindex}

\item {} 
\DUrole{xref,std,std-ref}{search}

\end{itemize}


\renewcommand{\indexname}{Python Module Index}
\begin{theindex}
\def\bigletter#1{{\Large\sffamily#1}\nopagebreak\vspace{1mm}}
\bigletter{s}
\item {\texttt{src}}, \pageref{src:module-src}
\item {\texttt{src.clustering}}, \pageref{src.clustering:module-src.clustering}
\item {\texttt{src.clustering.cluster\_mappings}}, \pageref{src.clustering:module-src.clustering.cluster_mappings}
\item {\texttt{src.eval}}, \pageref{src.eval:module-src.eval}
\item {\texttt{src.eval.analogy}}, \pageref{src.eval:module-src.eval.analogy}
\item {\texttt{src.eval.eval\_vectors}}, \pageref{src.eval:module-src.eval.eval_vectors}
\item {\texttt{src.eval.word\_similarity}}, \pageref{src.eval:module-src.eval.word_similarity}
\item {\texttt{src.mapping}}, \pageref{src.mapping:module-src.mapping}
\item {\texttt{src.mapping.mapthreading}}, \pageref{src.mapping:module-src.mapping.mapthreading}
\item {\texttt{src.misc}}, \pageref{src.misc:module-src.misc}
\item {\texttt{src.misc.decorators}}, \pageref{src.misc:module-src.misc.decorators}
\item {\texttt{src.misc.helpers}}, \pageref{src.misc:module-src.misc.helpers}
\item {\texttt{src.prep}}, \pageref{src.prep:module-src.prep}
\item {\texttt{src.prep.corpus}}, \pageref{src.prep.corpus:module-src.prep.corpus}
\item {\texttt{src.prep.corpus.convert\_to\_plain}}, \pageref{src.prep.corpus:module-src.prep.corpus.convert_to_plain}
\item {\texttt{src.prep.corpus.extract\_conll}}, \pageref{src.prep.corpus:module-src.prep.corpus.extract_conll}
\item {\texttt{src.prep.corpus.mapper}}, \pageref{src.prep.corpus:module-src.prep.corpus.mapper}
\item {\texttt{src.prep.corpus.reducer}}, \pageref{src.prep.corpus:module-src.prep.corpus.reducer}
\item {\texttt{src.prep.nes}}, \pageref{src.prep.nes:module-src.prep.nes}
\item {\texttt{src.prep.nes.extractNE}}, \pageref{src.prep.nes:module-src.prep.nes.extractNE}
\item {\texttt{src.prep.nes.merge}}, \pageref{src.prep.nes:module-src.prep.nes.merge}
\item {\texttt{src.prep.nes.mwe}}, \pageref{src.prep.nes:module-src.prep.nes.mwe}
\item {\texttt{src.prep.nes.statistics}}, \pageref{src.prep.nes:module-src.prep.nes.statistics}
\item {\texttt{src.prep.relations}}, \pageref{src.prep.relations:module-src.prep.relations}
\item {\texttt{src.prep.relations.relations}}, \pageref{src.prep.relations:module-src.prep.relations.relations}
\item {\texttt{src.trans\_e}}, \pageref{src.trans_e:module-src.trans_e}
\item {\texttt{src.trans\_e.add\_inverse\_relations}}, \pageref{src.trans_e:module-src.trans_e.add_inverse_relations}
\item {\texttt{src.trans\_e.clean\_relations}}, \pageref{src.trans_e:module-src.trans_e.clean_relations}
\item {\texttt{src.trans\_e.contains\_entities}}, \pageref{src.trans_e:module-src.trans_e.contains_entities}
\item {\texttt{src.trans\_e.convert\_relations}}, \pageref{src.trans_e:module-src.trans_e.convert_relations}
\item {\texttt{src.trans\_e.differentiate\_datasets}}, \pageref{src.trans_e:module-src.trans_e.differentiate_datasets}
\item {\texttt{src.trans\_e.partition\_data}}, \pageref{src.trans_e:module-src.trans_e.partition_data}
\item {\texttt{src.trans\_e.trans\_we}}, \pageref{src.trans_e:module-src.trans_e.trans_we}
\end{theindex}

\renewcommand{\indexname}{Index}
\printindex
\end{document}
